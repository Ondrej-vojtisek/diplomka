%% Load document class fithesis2
%% {10pt, 11pt, 12pt}
%% {draft, final}
%% {oneside, twoside}
%% {onecolumn, twocolumn}
\documentclass[11pt,draft,oneside]{fithesis2}

%% Basic packages
\usepackage[czech]{babel}
\usepackage{cmap}
\usepackage[T1]{fontenc}
\usepackage{lmodern}
\usepackage[utf8]{inputenc}
\usepackage{graphicx}

%% Additional packages for colors, advanced
%% formatting options, etc.
\usepackage{color}
\usepackage{microtype}
\usepackage{url}
\usepackage{cslatexquotes}
\usepackage{fancyvrb}
\usepackage[small,bf]{caption}
\usepackage[plainpages=false,pdfpagelabels,unicode]{hyperref}
\usepackage[all]{hypcap}

%% Fix long URLs in DVIs
\usepackage{ifpdf}

\ifpdf
\else
  \usepackage{breakurl}
\fi

%% Packages used to generate various lists
\usepackage{makeidx}
\makeindex

\usepackage[xindy]{glossaries}
\makeglossary

%% Use STAR and CIRCLE signs for nested
%% itemized lists
\renewcommand{\labelitemii}{$\star$}
\renewcommand{\labelitemiii}{$\circ$}

%% Title page information
\thesistitle{Návrh a implementace centrálního indexu BBMRI\_CZ}
\thesissubtitle{Diplomová práce}
\thesisstudent{Bc. Ondřej Vojtíšek}
\thesiswoman{false} %% Important when using Slovak or Czech lang
\thesisfaculty{fi}  %% {fi, eco, law, sci, fsps, phil, ped, med, fss}
\thesislang{cs}     %% {en, sk, cs}
\thesisyear{Jaro 2014}
\thesisadvisor{RNDr. Petr Holub, Ph.D.}

%% Beginning of the document
\begin{document}

%% Front page with a logo and basic thesis information
\FrontMatter
\ThesisTitlePage

%% Thesis declaration (required)
\begin{ThesisDeclaration}
  \DeclarationText
  \AdvisorName
\end{ThesisDeclaration}

%% Thanks (optional)
\begin{ThesisThanks}
TODO: řešit později
\end{ThesisThanks}

%% Shrnutí
\begin{ThesisAbstract}
TODO: řešit později
\end{ThesisAbstract}


%% Abstract
\begin{ThesisAbstracten}
TODO: Zjistit jak zapsat abstract v angličtině
\end{ThesisAbstracten}



%% Keywords (required)
\begin{ThesisKeyWords}
BBMRI, BBMRI\_CZ, Java EE, Stripes, Spring, JPA
\end{ThesisKeyWords}

%% Beginning of the thesis itself
\MainMatter

%% TOC (required)
\tableofcontents

% ------------------------------------------------------------------------      
% Uvod
\chapter{Úvod}
TODO nakonec

% ------------------------------------------------------------------------      
% Analyza
\chapter{Analýza}

% ------------------------   
% Section 
\section{Popis projektu BBMRI\_CZ}
Popis projektu, účel a cíle. Záštita, zapojení institucí ERIC, RECAMO, CERIT-SC

% ------------------------   
% SubSection 
\subsection{Workflow}
Jaké bylo zamýšlené workflow systému. Projektová žádost, formální schválení projektu, žádost o vzorky, schválení sady vzorků,...diagram
definování hranic systému

% ------------------------   
% Section 
\section{Požadavky na IT infrastrukturu}

% ------------------------   
% SubSection 
\subsection{Zabezpečení a anonymita pacientů}

Anonymita - klinická vs. věděcká data
Uvažované varianty, možné problémy ke kterým by mohlo dojít - migrace pacientů, duplicita rodných čísel

% ------------------------   
% Section
\section{Zapojené instituce a jejich specifika}
Číselníky, identifikátory, apod



% ------------------------------------------------------------------------      
% Návrh
\chapter{Návrh}
datový model, class diagram, vysvětlení jednotlivých tříd

% ------------------------------------------------------------------------      
% Implementace
\chapter{Implementace}
použité technologie, diskuze nad použitými technologiemi

% ------------------------------------------------------------------------      
% Závěr
\chapter{Závěr}

%% Lists of tables and figures, glossary, etc.
\printindex
\printglossary
\listoffigures
\listoftables

%% Bibliography from references.bib
\begingroup
\def\tmpchapter{0}
\renewcommand{\chaptername}{}
\renewcommand{\thechapter}{}
\addtocontents{toc}{\setcounter{tocdepth}{-1}}
\chapter{Zdroje}
\renewcommand{\chapter}[2]{}% for other classes

\bibliographystyle{plain}
\bibliography{references}

\addtocontents{toc}{\setcounter{tocdepth}{2}}
\endgroup

%% Additional materials
\appendix

%% End of the whole document
\end{document}