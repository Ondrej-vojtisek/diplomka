
\documentclass[11pt, final, oneside]{fithesis2}

%% Basic packages
\usepackage[czech]{babel}
\usepackage[utf8]{inputenc}
\usepackage{cmap}
\usepackage[T1]{fontenc}
%\usepackage{lmodern}
\usepackage{graphicx}
\DeclareGraphicsExtensions{.pdf,.png,.jpg}
\graphicspath{ {./images/} }
%% Package used in architecture document
\usepackage{tikz}
\usetikzlibrary{%
  arrows,%
  fit,%
  patterns,%
  shapes.geometric,%
  shapes.misc,%
  shapes.symbols,%
  shapes.arrows,%
  shapes.callouts,%
  shapes.multipart,%
  shapes.gates.logic.US,%
  shapes.gates.logic.IEC,%
  er,%
  backgrounds,%
  chains,%
  trees,%
  matrix,%
  calendar,%
  folding,%
  fadings,%
  through,%
  positioning,%
  scopes,%
  decorations.fractals,%
  decorations.shapes,%
  decorations.text,%
  decorations.pathmorphing,%
  decorations.pathreplacing,%
  decorations.footprints,%
  decorations.markings,%
  shadows}  


%% Additional packages for colors, advanced
%% formatting options, etc.
\usepackage{color}
\usepackage{microtype}
\usepackage{url}
\usepackage{cslatexquotes}
\usepackage{fancyvrb}
\usepackage[small,bf]{caption}
\usepackage[numbers]{natbib} 
%%\usepackage[plainpages=false,pdfpagelabels,unicode]{hyperref}
\usepackage{amssymb} % required by correction notes
\usepackage{hyperref}
%\usepackage[all]{hypcap} - problem s nutnosti mit caption u figure
\usepackage{xspace}
\usepackage{paralist} % CompactItem
\usepackage{fancyvrb} % Verbatim zarovnany na stred
%\usepackage[paper=a4paper,top=2.5cm,bottom=2.5cm,left=2.5cm,right=2.5cm,foot=1cm]{geometry} % Nastavení rozměrů stránky

% vypis teplot
\usepackage{mathpazo}
\renewcommand{\ttdefault}{txtt}
\usepackage{siunitx}

% column layout
\usepackage{multicol}

\newcommand{\polozka}[1]{\item {\bf #1}\xspace}
\newcommand{\paragraphNewLine}[1]{\paragraph*{#1}\mbox{}\\}

\usepackage{listings} % Code examples
%% Nove caption u listingu kódu
\renewcommand\lstlistingname{XML schéma} 
%% XML Listing - http://tex.stackexchange.com/questions/10255/xml-syntax-highlighting
\usepackage{color}
\usepackage{textcomp}
\definecolor{gray}{rgb}{0.4,0.4,0.4}
\definecolor{darkblue}{rgb}{0.0,0.0,0.6}
\definecolor{cyan}{rgb}{0.0,0.6,0.6}
\definecolor{palatinatepurple}{rgb}{0.41, 0.16, 0.38}

\lstset{
	captionpos=b, % Caption je pod ukazkou kodu
  basicstyle=\ttfamily\scriptsize,
  columns=fullflexible,
  showstringspaces=false,
  commentstyle=\color{gray}\upshape,
	breaklines=true
}

\lstdefinelanguage{XML}
{
%% XML
%%	comment=[l]{##}
  morestring=[b]",
  morestring=[s]{>}{<},
  morecomment=[s]{<?}{?>},
  stringstyle=\color{palatinatepurple},
  identifierstyle=\color{darkblue},
  keywordstyle=\color{cyan},
  morekeywords={xmlns, version, type, element, attribute, default namespace}% list your attributes here
}

\widowpenalty 10000
\clubpenalty 10000

% Komentáře pomocí \ovnote{text}
\newcounter{ovNoteCounter}
\newcommand{\ovnote}[1]{{\scriptsize\color{red} $\divideontimes$ \refstepcounter{ovNoteCounter}\textsf{[OV]$_{\arabic{ovNoteCounter}}$:{#1}}}}

%% Fix long URLs in DVIs
\usepackage{ifpdf}

\ifpdf
\else
  \usepackage{breakurl}
\fi

%% Packages used to generate various lists
\usepackage{makeidx}
\makeindex

\usepackage[xindy]{glossaries}
\makeglossary

%% Use STAR sign for nested
%% itemized lists
\renewcommand{\labelitemii}{$\star$}
\newcommand{\ProjectName}{\mbox{BBMRI\_CZ}\xspace}

%% Title page information
\thesistitle{Návrh a~implementace centrálního indexu \ProjectName}
\thesissubtitle{Diplomová práce}
\thesisstudent{Ondřej Vojtíšek}
\thesiswoman{false} %% Important when using Slovak or Czech lang
\thesisfaculty{fi}  %% {fi, eco, law, sci, fsps, phil, ped, med, fss}
\thesislang{cs}     %% {en, sk, cs}
\thesisyear{Jaro 2014}
\thesisadvisor{RNDr. Petr Holub, Ph.D.}



%% Beginning of the document
\begin{document}


%% Nastavení zalamování slov
\hyphenation{in-fra-struk-tu-ra in-fra-struk-tu-ry we-bo-vé-ho iden-ti-fi-ká-toru žá-dan-ky
me-ta-s-tá-zy ali-kvo-tu}

%% Front page with a logo and basic thesis information
\FrontMatter
\ThesisTitlePage

%% Thesis declaration (required)
\begin{ThesisDeclaration}
  \DeclarationText
  \AdvisorName
\end{ThesisDeclaration}

%% Thanks
\begin{ThesisThanks}
Na tomto místě bych rád poděkoval vedoucímu mé práce RNDr. Petru Holubovi, PhD., za vedení a~množství rad jak odborných, tak i~zcela neinformatických. Vážím si toho, že mi bylo umožněno pracovat na reálném a~smysluplném projektu, přestože jsem do něho vstupoval takřka jako \textit{tabula rasa}.

Mé poděkování též patří Mgr. Janu Sochorovi za velkou pomoc s~pronikáním do tajů webového vývoje. 
V~neposlední řadě bych rád poděkoval kolegům a~partnerům zapojeným do projektu za jejich trpělivost a~ochotu pomoci. Z~mnoha osob bych zde rád jmenoval alespoň Ing. Danu Knoflíčkovou z~Masarykova onkologického ústavu. 
\end{ThesisThanks}




% ------------------------------------------------------------------------   
% ------------------------------------------------------------------------   
%% Shrnutí
% ------------------------------------------------------------------------   
\chapter*{Shrnutí}
Diplomová práce se zaměřuje na analýzu, návrh a~implementaci informatické infrastruktury pro projekt \ProjectName. Projekt se zabývá podporou výzkumu nádorových onemocnění v~České republice. Informatická infrastruktura má figurovat jako nástroj usnadňující používat pro výzkum biologický materiál jiných institucí. Práce obsahuje charakteristiku systémů, se kterými bude aplikace spolupracovat, a~představuje řešení splňující požadavky projektu.

\vspace{4em} 
\noindent {\Large\textbf{Klíčová slova}} \\ \\ 
\noindent BBMRI, \ProjectName, BBMRI-ERIC, Java EE, Stripes, Spring, Hibernate, biobanking

%% Beginning of the thesis itself
\MainMatter

%% TOC (required)
\tableofcontents

% ------------------------------------------------------------------------   
% ------------------------------------------------------------------------      
% Uvod
% ------------------------------------------------------------------------   
\chapter{Úvod}

Nelehkou částí výzkumu v~oblasti medicíny je fáze předcházející samotnému výzkumnému procesu, tj. sběr požadovaných dat. Oproti jiným oblastem, kde lze dostatek dat vygenerovat, naměřit nebo zakoupit, je tato oblast daleko více svázána a~omezena jak legislativou, tak i~s~množstvím ryze praktických překážek.

Obecně lze říci, že dat musí být dostatečné množství, aby z~nich bylo možné vyvodit relevantní závěry. Data musí být konzistentní, tj. musí být získaná za srovnatelných nebo alespoň souměřitelných podmínek, a~musí být získána způsobem respektujícím právní předpisy. V~praxi je kladeno na původce dat několik požadavků, které často není snadné naplnit:

\begin{itemize}
\polozka{Struktura ukládaných dat} -- V~nemocničních systémech bývá častým zvykem ukládát velké množství dat ve formě volného textu. Nemocnice může mít v~evidenci tisíce pacientů, ale využití těchto dat je možné až po převedení dat do strukturované podoby. To mnohdy nelze udělat jinak než manuálním přepisem.
	
	\polozka{Množina ukládaných dat} -- Zdravotnická zařízení mohou používat informační systém jiné verze, jiného typu nebo úplně od jiného výrobce. Z~toho plyne, že se může lišit datový model a~to jak syntakticky, tak i~semantikou uložených dat. Kupříkladu existuje množství různých číselníků a~klasifikací, takže informace stejného významu může být strukturovaně uložena několika způsoby. Pro agregaci dat je třeba se ujednotit na tom co má být ze systému exportováno a~v~jakém formátu. 
		
	\polozka{Metodika sběru dat} -- Musí být zcela jasné, jakým způsobem byla data získána. Společně s~daty je potřeba exportovat i~metodiku sběru dat (někdy nazývanou jako protokol). Naplnění této podmínky je obzvlášť důležité při agregaci dat ze zahraničních zdrojů. Postup považovaný za samozřejmý se v~jiném státě může řešit zcela jinou metodou (rozdílné fyzikální jednotky, kulturně-sociální vývoj, atd.). Ani v~kontextu jednoho státu to ale nelze opomíjet. 
	
	\polozka{Oprávnění k~využít dat pro výzkum} -- Pacientská data byla získána primárně pro potřeby stanovení optimální léčby pacienta. Jakékoli jiné využití je stanoveno výjimkami v~zákoně nebo podléhá souhlasu pacienta. Souhlas může být udělen na konkrétní výzkum nebo obecně na veškerý výzkum.
\end{itemize}

Příkladem evropských projektů, které se zabývají budováním platforem umožňujícím vzájemnou výměnu dat pro výzkumné účely, jsou ELIXIR\footnote{http://www.elixir-europe.org/} nebo THALAMOSS\footnote{http://www.thalamoss.eu/}. 

\ProjectName je v~ČR realizovaný projekt, který si klade podobné cíle jako výše uvedené projekty. Cílem je vybudovat infrastrukturu, která usnadní vývoj v~oblasti léčby nádorových onemocnění. Vyměňovanými daty jsou zde jak informace, tak i~fyzické vzorky biologického materiálu odebraného pacientům v~průběhu léčby. Pokud pacient souhlasil s~účastí, je část materiálu odebraného pro diagnosticko-terapeutické\footnote{Tj. běžné nezbytné odběry v~průběhu léčby.} účely uskladněna pro využití na budoucí výzkum. \uv{Skladům} biologického materiálu se říká biobanky. 
Tato práce se zabývá informatickou částí projektu \ProjectName a~v~následujícím textu je popsáno, jakým způsobem je pro tento konkrétní projekt vyřešen sběr a~distribuce dat. 

Text práce se dělí na tři části: Analýzu, Návrh a~Implementaci. V~kapitole Analýza je popsán projekt, aktuální stav a~požadavky na informatickou infrastrukturu. Kapitola vychází z~dokumentu \cite{ARCH_2014_1_25}, kterého jsem spoluautorem. Kapitoly Návrh a~Implementace jsou již zcela mojí prací. V~kapitole Návrh je popsána navržená architektura systému na základě vstupních požadavků. Kapitola Implementace obsahuje stručný popis použitých technologií a~popis některých zajímavějších implementačních detailů. Při psaní jsem kladl důraz především na zaznamenání skutečností, které mohou posloužit budoucím vývojářům aplikace \ProjectName.

% ------------------------------------------------------------------------ 
% ------------------------------------------------------------------------      
% Analyza
% ------------------------------------------------------------------------   
\chapter{Analýza}\label{chapter:analysis}

% ------------------------------------------------------------------------   
% ------------------------------------------------------------------------   
\section{Popis projektu \ProjectName}\label{chapter:analysis:section:projectDescription}
BBMRI (Biobanking and Biomolecular Resources Research Infrastructure) je celoevropský projekt, jehož cílem je vytvořit jednotnou infrastrukturu nad zdravotnickými zařízeními, biobankami a~dalšími výzkumnými pracovišti, umožňující výměnu dat a~biologického materiálu mezi těmito institucemi pro potřeby výzkumu. Dílčími cíli projektu je vyřešení legislativních otázek, týkajících se nakládání s~biologickým materiálem, standardizace uchovávání vzorků, sjednocení pohledu na strukturu ukládaných dat a~další kroky nutné pro podporu výzkumu a~spolupráce napříč pracovišti a~zapojenými státy. Project výzkumné infrastruktury BBMRI je implementován v~rámci konsorcia ERIC (European Research Infrastructure Consortium), proto je od roku 2013 projekt označován jako BBMRI-ERIC.

\ProjectName je česká odnož projektu BBMRI, jejímž cílem je vytvořit českou síť biobank přidružených k~lékařským fakultám, které budou dlouhodobě uchovávat biologický materiál pacientů s~nádorovými onemocněními. Cílem projektu je, stejně jako v~případě celoevropského BBMRI, prostřednictvím výměny uchovávaných vzorků a~souvisejících informací mezi institucemi zlepšit prostředí pro výzkum nádorových onemocnění. V~dlouhodobém horizontu je žádoucí zapojit vznikající infrastrukturu \ProjectName do celoevropské infrastruktury BBMRI-ERIC. 

Koordinátorem projektu je Masarykův onkologický ústav (dále MOÚ). Technologickým partnerem na projektu je centrum CERIT-SC. To má na starosti vybudování a~správu informatické infrastruktury – indexové a~monitorovací služby \ProjectName. Návrhu a~implementaci těchto služeb se věnuje tato práce.

% ------------------------------------------------------------------------   
% ------------------------------------------------------------------------   
\section{Indexová služba}\label{chapter:analysis:section:index}
Pacientská data jsou uložena v~nemocničním informačním systému (dále jen NIS) a~obsahují veškeré záznamy o~průběhu léčby pacienta a~o~souvisejících vyšetřeních, tj. operace, laboratorní analýzy, kontroly a~další záznamy. Indexová služba \ProjectName (zkráceně index) představuje centrální bod infrastruktury projektu, do kterého jsou ze všech NISů nahrána data popisující uložený biologický materiál pacientů. Index je pouze databází a~nejedná se o~faktické úložiště vzorků.

Biobanka se organizačně dělí na modul pro dlouhodobé uložení vzorků (dále LTS z~anglického Long Term Storage) a~modul pro krátkodobé uložení vzorků (dále STS z~anglického Short Term Storage). 

Vzorky v~STS si lze představit jako pravidelně odebíraný materiál při každé kontrole pacienta (např. krev, moč), na kterém lze pozorovat určitý vývoj choroby nebo léčby. Jelikož se jedná o~velké množství vzorků (pro každého pacienta řádově desítky, celkově pak desítky tisíc vzorků), jsou tyto vzorky z~ekonomických důvodů uchovávány jen po omezenou dobu. Praxe na MOÚ je uchovávat vzorky v~STS po dobu jednoho roku.

Vzorky v~LTS si lze představit jako jednorázově pořízený materiál, získaný např. při operaci pacienta (např. uchovávaná tkáň). 
Těchto vzorků je řádově menší množství oproti STS a~proto~mohou být v~biobance uloženy tak dlouho jak je třeba.

\begin{figure}[htbp]
\centering

\begin{tikzpicture}[
line width=1mm,
node distance = 7mm,
linka/.style = {semithick},
sipka/.style = {-stealth',semithick},
hranice/.style = {dotted},
nadpis/.style = {text width=100mm,text centered},
polozkadb/.style = {draw,semithick,text width=30mm,text badly centered},
pacient/.style = {polozkadb,fill=blue!20},
STS/.style = {polozkadb,fill=red!20},
STSitem/.style = {STS,fill=red!10,anchor=west},
LTS/.style = {polozkadb,fill=green!20},
LTSitem/.style = {LTS,fill=green!10,anchor=west},
lecba/.style = {text width=30mm,text badly centered},
]

\draw node[pacient] (PacientID) {Pacient (rodné~číslo)};
\draw node[STS, below right = 10mm and 5mm of PacientID.center] (STS) {Krátkodobé úložiště (STS)};
\draw node[STSitem, below right = 10mm and 5mm of STS.center, anchor=north west] (STSserum) {sérum (rezerva)};
\draw node[STSitem, below = 4mm of STSserum.south west, anchor=north west] (STSplasma) {plasma};
\draw node[STSitem, below = 4mm of STSplasma.south west, anchor=north west] (STSurine) {moč};
\draw node[LTS, right = 25mm of STS.center] (LTS) {Dlouhodobé úložiště (LTS)};
\draw node[LTSitem, below right = 10mm and 5mm of LTS.center, anchor=north west, text width=35mm] (LTStkan) {tkáň + diagnostická klasifikace};
\draw node[LTSitem, below = 4mm of LTStkan.south west, anchor=north west] (LTSdna) {genomová DNA, plná krev};
\draw node[LTSitem, below = 4mm of LTSdna.south west, anchor=north west] (LTSrna) {RNA};
\draw node[LTSitem, below = 4mm of LTSrna.south west, anchor=north west] (LTSserum) {sérum};
\draw node[LTSitem, below = 4mm of LTSserum.south west, anchor=north west] (LTSplasma) {plasma};
\draw node[LTSitem, below = 4mm of LTSplasma.south west, anchor=north west] (LTSurine) {moč};

\bgroup\shorthandoff{-}
\draw[linka] (PacientID) -| (STS);
\draw[linka] (PacientID) -| (LTS);
\draw[linka] (STS) |- (STSserum);
\draw[linka] (STS) |- (STSplasma);
\draw[linka] (STS) |- (STSurine);
\draw[linka] (LTS) |- (LTStkan);
\draw[linka] (LTS) |- (LTSdna);
\draw[linka] (LTS) |- (LTSrna);
\draw[linka] (LTS) |- (LTSserum);
\draw[linka] (LTS) |- (LTSplasma);
\draw[linka] (LTS) |- (LTSurine);
\egroup
\end{tikzpicture}
\caption{Struktura biobanky~\cite{ARCH_2014_1_25}.}
\label{fig:index:bb-struktura}

\end{figure}

% ------------------------------------------------------------------------   
% ------------------------------------------------------------------------   
\section{Monitorovací služba}\label{chapter:analysis:section:monitoring}
Vzorky jsou v~biobance uloženy dlouhodobě, aby bylo možné je laboratorně analyzovat i~delší dobu po jejich odebrání. Pro zachování konzistence výstupů laboratorních testů je vysoce žádoucí skladovat vzorky za konstantních podmínek s~možností detekovat porušení podmínek uložení. K~tomu slouží monitorovací služba \ProjectName.

Vzorky v~biobankách jsou uchovávány v~mrazu. Pro posouzení řádného skladování je klíčové sledovat teplotu vzduchu uvnitř úložiště a~především teplotní extrémy. Příklad infrastruktury uložení bude ilustrován na brněnské onkologii. 

Na MOÚ používají pro uložení materiálu při vyšších teplotách lednice vzhledově podobné zařízením z~kuchyňského prostředí. Jejich provozní teplota se pohybuje okolo \SI{-40}{\celsius} až \SI{-20}{\celsius}. Pro uložení při nižších teplotách jsou použity tzv. Dewarovy nádoby\footnote{Dewarova nádoba~\cite{dewar} je zařízení konstrukčně podobné termosce (tj. vakuem izolovaná nádoba), jen s~tím rozdílem, že nemá pevně uzavřené víko (resp. má záměrně netěsnící víko). Dovnitř se nalije zkapalněný plyn (např. dusík), který se postupně vypařuje. Nad hladinou tekutiny, v~parách, je uskladněn chlazený materiál. Víko není utěsněno, aby nádoba nebyla roztržena přetlakem.}. Pro zajištění optimálních podmínek skladování (tj. požadované teploty) je nutné sledovat teplotu vzduchu a~výšku hladiny kapalného dusíku. Při poklesu hladiny pod určitou mez (vlivem vypaření) je nutno kapalný dusík doplnit. 

Projekt \ProjectName klade na zapojené instituce nárok na zajištění kvality uskladněného materiálu. Z~toho vyplývá požadavek, aby do ISu projektu byly exportovány i~hodnoty naměřené teploty ve skladovací infrastruktuře.

V~situaci, kdy monitoring teploty indikuje nevhodné podmínky uložení, je nutné v~systému rozpoznat, které vzorky byly touto událostí ovlivněny. Z~toho důvodu je nutné do systému importovat i~informaci o~místě uskladnění vzorků.

% ------------------------------------------------------------------------   
% ------------------------------------------------------------------------   
\section{Příklady užití systému a~popis souvisejícího workflow}\label{chapter:analysis:section:workflow}
Hlavním aktérem systému (dále též uživatel) je výzkumný pracovník pracující na projektu v~oblasti výzkumu nádorových onemocnění. Pro realizaci výzkumu potřebuje provést sadu laboratorních testů na relevantním počtu biologických vzorků určitých parametrů. Domovská instituce výzkumníka často nemá potřebné vzorky, a~nebo jich nemá dostatek a~výzkumník pro realizaci výzkumného záměru potřebuje materiál i~z~jiné instituce. Jednotlivým částem tohoto scénáře se věnují následující kapitoly.

\begin{figure}[h!]
\centering
\begin{tikzpicture}[
node distance = 15mm,
sipka/.style = {-stealth',semithick},
faze/.style = {draw,fill=black!10,semithick},
kontejner/.style = {draw, densely dashed, anchor = north east, inner sep = 3mm},
popiskontejneru/.style = {anchor = west, font = \em, text badly ragged},
]
\draw node[faze] (NIS) {NIS (+ další zdroje)};
\draw node[popiskontejneru, above = 3mm of NIS.north west, xshift = 8mm] (kMajitel) {Nemocnice};
\draw node[kontejner, fit = (NIS) (kMajitel)] {};

\draw node[faze, below = 30.5mm of NIS, text width = 8cm, text badly centered] (storage) {Navázání informací o~vzorku na uložení v~kontejneru (není-li součástí NIS)};
\draw node[faze, below of = storage] (export) {Převod dat do exportního formátu};
\draw node[popiskontejneru, above = 3mm of storage.north west, xshift = 20mm, text width = 4cm] (kExporter) {Nemocnice nebo partnerská biobanka hostující vzorky};
\draw node[kontejner, fit = (storage) (export) (kExporter)](kBiobank) {};

\draw node[faze, left = 15mm of kBiobank.west, rotate=90, xshift = 22mm](samples){Seznam žádaných vzorků};

\draw node[faze, below = 25mm of export] (bbidx) {Uložení dat v~indexu \ProjectName};
\draw node[popiskontejneru, above = 3mm of bbidx.north west, xshift = 20mm, text width = 4cm] (kCentral) {Centrální infrastruktura \ProjectName};
\draw node[kontejner, fit = (bbidx) (kCentral)] {};

\draw[sipka] (NIS) -- (storage);
\draw[sipka] (storage) -- (export);
\draw[sipka] (export) -- (bbidx);
\draw[sipka,dotted] (bbidx.west) .. controls +(left:3cm) and +(left:0cm)  .. (samples.west);
\draw[sipka,dotted] (samples.east) .. controls +(up:1cm) and +(left:3cm)  .. (NIS.west);


\end{tikzpicture}
\caption{Schéma předávání dat o~vzorcích do indexové služby \ProjectName~\cite{ARCH_2014_1_25}.}
\label{fig:bbidx:data-acquisition}
\end{figure}


% ------------------------------------------------------------------------     
\subsection{Založení projektu}
Uživatelé mohou žádat o~biologický materiál, pouze pokud mají existující a~v~systému akceptovaný projekt. Index slouží pouze jako evidence projektů, aby bylo jasné, s~jakým mandátem uživatel o~vzorky žádá. Index tedy nesupluje grantové nadace ani jiné role v~řetězci života projektu. 

Od projektu je očekáváno, že má vyřešené financování, je zastřešen důvěryhodnou institucí a~schválen etickou komisí\footnote{Komise posuzující etické hledisko projektů biomedicínského výzkumu, zřízená v~rámci instituce, kde je projekt realizován.} dané instituce. Jediný dokument, který je po uživateli explicitně žádán navíc, je tzv. \textit{Material Transfer Agreement} (dále jen MTA), podrobněji popsaný v~kapitole~\ref{chapter:analysis:section:legal}. Uživatel nahraje tato data do systému, kde jsou formálně zkontrolovány správcem systému. Pokud jsou všechny formální náležitosti splněny, je projekt akceptován a~může být v~rámci jeho realizace žádáno o~vzorky.
Uživatel s~akceptovaným projektem je nazýván oprávněnou osobou. 

% ------------------------------------------------------------------------     
\subsection{Žádost o~vzorky}

Žadatel formou nestrukturovaného textu formuluje, o~jaké vzorky má zájem, a~tuto žádost odešle konkrétní biobance. Správce této biobanky (patolog) na základě slovního popisu vybere příslušnou sadu vzorků. Seznam žádaných vzorků (včetně jejich počtu a~místa, kde se nachází) se vygeneruje laboratorním pracovníkům, kteří připraví sadu fyzických vzorků k~předání. V~NISu nemocnice vydávající vzorky budou následně upraveny počty vzorků a~aktualizovaná data budou nahrána do centrálního indexu při následujícím importu (viz obrázek ~\ref{fig:bbidx:data-acquisition}). Vydány mohou být pouze alikvoty\footnote{Alikvotem je v~kontextu práce myšlen díl vzorku odebraného biologického materiálu, např. tenký \uv{plátek} vyoperovaného nádoru.} označené jako vydatelné. Kromě těch si biobanka v~úložišti typicky drží i~část alikvotů pro servisní účely, které nejsou poskytované pro výzkum.

Specifikace žádaných vzorků formou nestrukturovaného textu vznikla na základě explicitního požadavku správců biobank. Ti nechtějí, aby oprávněné osoby měly přístup k~parametrickému prohledávání databáze vzorků -- což implementovala i~jedna z~průběžných verzí aplikace. Vedení projektu se rozhodlo tuto cestu prozatím opustit.

Žádost o~vzorky není nároková a~to ani v~situaci, kdy projekt splnil veškeré náležitosti a~žádá o~vzorky biobanku, která žádanými vzorky disponuje. Tato skutečnost opět vychází z~požadavků správců biobank.

V~situaci, kdy je uživatel realizující projekt současně i~správcem biobanky a~podává žádost u~své domovské instituce, je povoleno, aby si žádost sám schválil.

% ------------------------------------------------------------------------    
\subsection{Samožádanky}

\begin{figure}[htb]
\centering
\begin{tikzpicture}[
node distance = 15mm,
sipka/.style = {-stealth',semithick},
faze/.style = {draw,fill=black!10,semithick},
]
\draw node[faze] (bbidx) {Oprávněný uživatel \ProjectName};
\draw node[faze, below of = bbidx] (req) {Vygenerování žádosti};
\draw node[faze, below of = req] (req2) {Biobanka 2};
\draw node[faze, below of = req2,text width=3cm,text badly centered] (ack2) {Přiřazení vzorků\\Schválení biobankou 2};
\draw node[faze, left = 2cm of req2] (req1) {Biobanku 1};
\draw node[faze, below of = req1,text width=3cm,text badly centered] (ack1) {Přiřazení vzorků\\Schválení biobankou 1};
\draw node[faze, right = 2cm of req2] (req3) {Biobanka 3};
\draw node[faze, below of = req3, text width=3cm,text badly centered] (ack3) {Zamítnutí biobankou 3};
\draw node[faze, below left = 15mm and 5mm of ack2.center] (recv) {Získání vzroků z~biobank 1 a 2};
\draw node[faze, below right = 15mm and 5mm of ack2.center] (archiv) {Archivace žádostí a rozhodnutí};

\draw[sipka] (bbidx) -- (req);
\draw[sipka] (req) -- (req1);
\draw[sipka] (req) -- (req2);
\draw[sipka] (req) -- (req3);
\draw[sipka] (req1) -- (ack1);
\draw[sipka] (req2) -- (ack2);
\draw[sipka] (req3) -- (ack3);
\draw[sipka] (ack1) -- (recv);
\draw[sipka] (ack2) -- (recv);
\draw[sipka] (ack1) -- (archiv);
\draw[sipka] (ack2) -- (archiv);
\draw[sipka] (ack3) -- (archiv);

\end{tikzpicture}
\caption{Schéma práce s~daty z~pohledu uživatele \ProjectName~\cite{ARCH_2014_1_25}.}
\label{fig:bbidx:user-interaction}
\end{figure}

Druhý scénář výběru vzorků se týká tzv. samožádanek. Jedná se o~situaci, kdy laboratorní pracovník potřebuje určitý vzorek biologického materiálu uloženého ve skladovací infrastruktuře, aby na něm provedl kontrolní testy s~cílem ověřit jeho kvalitu. Od toho se pak odvozuje kvalita obdobně uloženého materiálu. 

Výdej takového typu se týká pouze vzorků uložených ve \uv{vlastní biobance} a~nevyžaduje schválení ani založení projektu. Samožádankou je možné žádat i~o~tzv. nevydatelné vzorky. 


% ------------------------------------------------------------------------     
\subsection{Rezervace}

Možnost rezervace reflektuje situaci, kdy žadatel má vymyšlený projektový záměr, ale zatím ještě nebyly uskutečněny všechny formální kroky předcházející realizaci projektu. Pro tuto situaci je definována možnost rezervace, prostřednictvím které je uživateli dovoleno alokovat vzorky ještě před zadáním projektu do systému. Výběr vzorků probíhá formou nestrukturovaného textu stejně jako u~žádosti o~vzorky. Rezervace se od žádosti liší jen tím, že není vázána na projekt a~má časově omezenou platnost.

Rezervace samotná pro vydání vzorků nestačí. Poskytuje ale uživateli jistotu, že pokud do určité doby podá projekt, bude mít potřebný biologický materiál. Pokud nestihne projekt podat do stanovené lhůty, alokace vzorků bude zrušena.

% ------------------------------------------------------------------------   
% ------------------------------------------------------------------------   
\section{Zabezpečení a~anonymita pacientů}
Zásadní rozdíl z~hlediska anonymity mezi NISy a~systémy jako je např. index \ProjectName je v~tom, že nemocniční systémy pracují s~klinickými daty s~cílem léčit konkrétního pacienta\footnote{Oprávnění pro přístup ke zdravotnické dokumentaci pacienta stanovuje zákon č. 260/2001 Sb., § 67b, ods. 10.}, zatímco \ProjectName má čistě výzkumný cíl. Ke klinickým datům pacienta má přístup lékař (nebo skupina lékařů), vázaný povinností mlčenlivosti\footnote{Podle § 51 zákona č. 372/2011 Sb., o~zachování mlčenlivosti v~souvislosti se zdravotními službami.} a~povinností konat pro dobro konkrétního pacienta (tedy léčit jej), a~NIS mu (nebo jim) poskytuje veškeré dostupné informace pro co nejsprávnější rozhodnutí o~dalším postupu léčby. Naproti tomu výzkumná data neslouží pro léčbu konkrétního pacienta, nevztahuje se na ně povinnost mlčenlivosti, a~nakládání s~mimi je podmíněno souhlasem pacienta. Souhlas definuje, k~čemu mohu být data použita a~jak s~nimi bude naloženo. Z~pohledu anonymity pak v~množině takových dat nesmí být umožněno jednoznačně rozpoznat identitu libovolného pacienta. 

% ------------------------------------------------------------------------   
\subsection{Identifikace pacientů a~související komplikace}
Přestože identita pacienta nesmí být prozrazena, je nezbytné pacienta jednoznačně identifikovat v~rámci \ProjectName, aby nemohlo dojít k~záměně záznamů (vzorků) od více různých pacientů. 

Za dobu realizace projektu bylo diskutováno několik způsobů řešení~\cite{ARCH_2014_1_25} jako např. využití externí anonymizační služby (drahé) nebo použití jednosměrné hashovací funkce na kombinaci rodného čísla, jména, příjmení a~soli\footnote{Pokud má útočník k~dispozici databázi obsahující jména, příjmení a~RČ, je útok hrubou silou na sebelepší hashovací funkci otázkou minut. Hash by bylo nutné \uv{posílit} technikou tzv. solení.} (nedostačující pro Úřad pro ochranu osobních údajů). 
Výsledkem je kompromis, kdy je pacient identifikován interním identifikátorem instituce, ve které byl léčen, doplněným o~unikátní prefix, tak aby se zamezilo duplicitám napříč institucemi.

Ze zvoleného přístupu plyne riziko, že identita pacienta bude prozrazena zaměstnancem nemocnice, ze které pochází pacientská data. Druhou negativní skutečností, kterou řešení není schopno pokrýt, je případ pacienta postupně léčeného ve více různých nemocnicích. Systém v~takovém případě bude chápat tyto záznamy jako data dvou různých pacientů namísto toho, aby je správně spojil dohromady. Obě tato negativa byla koordinátory projektu vyhodnocena jako akceptovatelné \uv{menší zlo}.

% ------------------------------------------------------------------------   
\subsection{Autentizace a~autorizace}\label{chapter:analysis:subsection:authorization}
Autentizace uživatelů bude implementována pomocí federalizované autentizační infrastruktury eduId\footnote{\url{http://www.eduid.cz} -- Česká akademická federace indentit, kterou spravuje sdružení CESNET.}. Pro uživatele, kteří nespadají pod žádného poskytovatele identit (IdP -- Identity Provider), je zde možnost využít tzv. Hostel\footnote{\url{http://www.hostel.eduid.cz} -- Služba poskytovaná sdružením CESNET v~rámci eduId, pro uživatele institucí, nezapojených do federace.}. 

\paragraphNewLine{Autorizace pro přístup do systému}
Autorizovány k~přístupu do systému jsou osoby, které mají zaměstnanecký poměr\footnote{Formu pracovně-studijního vztahu uživatele k~instituci popisuje atribut \textit{affiliation} a~konkrétně hodnota \textit{@employee} definuje, že uživatel je zaměstnancem.} v~instituci, s~jejímž loginem se autentizují.

\paragraphNewLine{Požadavky na autorizaci operací}
Přistupovat k~monitoringu biobank je oprávněn uživatel, který se podílí minimálně na jednom projektu evidovaném a~akceptovaném v~indexu.
Vydávat vzorky (tj. připravovat sady vzorků pro vydání) může pouze zodpovědná osoba z~příslušné biobanky, kde je vzorek uchováván.

% ------------------------------------------------------------------------   
% ------------------------------------------------------------------------   
\section{Právní otázky}\label{chapter:analysis:section:legal}
Souhlas pacienta (nebo také informovaný souhlas -- informed consent) je nutným dokumentem opravňujícím k~využívání pacientských dat pro potřeby výzkumu. Takový dokument pacienta informuje o~povaze výzkumu, rizicích a~způsobu nakládání s~jeho osobními údaji. Obsah dokumentu je definován zákonem, ale jeho podoba se mezi institucemi liší. 

Druhým klíčovým dokumentem pro work-flow je Souhlas s~uchováním a~použitím nevyužitých zbytků ze vzorků získaných z~těla pacienta\footnote{Souhlas vychází z~§ 81 odst. 2 písm. a) bod 1. zákona č. 372/2011 Sb., o~zdravotních službách.}. Ten stanovuje, že část těla odebranou pacientovi při poskytování zdravotní péče lze uchovat a~použít pro potřeby vědy. 

Index \ProjectName z~povahy vědeckého účelu může zpracovávat pouze a~jedině data pacientů s~informovaným souhlasem a~s~podepsaným dokumentem o~možnosti nakládat se zbytky materiálu. Rozhodnutí o~exportu pacienta je ponecháno na straně nemocnic a~v~indexu se předpokládá, že pro každého pacienta existují příslušné podepsané dokumenty v~nemocničním archivu.
Při exportu pacientských dat je nutné explicitně uvést, že pacient souhlasil. Na základě souhlasu v~exportech se očekává, že souhlas je uložen a~dohledatelný  v~instituci, odkud pochází pacientská data.

Nutný dokument pro zadání projektu do indexu \ProjectName je MTA, který definuje, jakým způsobem je dovoleno s~materiálem pracovat, na co je možné jej použít, co s~ním dělat po naplnění projektového záměru a~jak v~případných výstupech projektu informovat o~původu dat. Kromě dalších bodů je také stanoven zákaz poskytnout získaný (byť třeba nevyužitý) materiál jakékoli třetí straně.


% ------------------------------------------------------------------------   
% ------------------------------------------------------------------------   
\section{Zapojené instituce a~jejich specifika}\label{sec:instituce}
Pro navržení centrálního indexu bylo v~první řadě nutné zjistit, jak vypadají ISy partnerů zapojených do projektu, s~jakými daty pracují, jaké používají číselníky a~jak je možné se k~nim připojit, čemuž se věnuje tato kapitola.

Je důležité zde rozlišovat vlastníka a~poskytovatele dat, což je typicky nemocnice, od správce biobanky, kterou je obvykle fakulta při univerzitě. Situace je odlišná pouze na MOÚ, kde je instituce zároveň vlastníkem dat i~správcem biobanky.

% ------------------------------------------------------------------------    
\subsection{Masarykův onkologický ústav}
Na MOÚ využívají NIS GreyFox\footnote{NIS GreyFox vytvořil RNDr. Alexandr Fuchs. Od roku 2008 je systém provozovaný společností STAPRO s.r.o.~\cite{GreyFox}}. Systém je strukturován do následujících modulů: tkáňový, sérový, genomový, bioptický\footnote{Bioptický modul popisuje odběr materiálu -- tj. informaci o~operaci, datu operace, operatérovi, který výkon provedl atd. Bioptický modul je definován tzv. bioptickou žádankou.} a~laboratorní. V~biobance je uloženo řádově desetitisíce vzorku.

MOÚ figuruje v~projektu jako koordinátor, proto byla velká část datového modelu převzata ze systému používaného na brněnské onkologii. Interní číselník používaný na MOÚ pro klasifikaci materiálu se nechází v~příloze~\ref{tab:ciselnik-mat-muni}.

% ------------------------------------------------------------------------   
\subsection{1. Lékařská fakulta Univerzity Karlovy v~Praze}

Na 1.~Lékařské fakultě Univerzity Karlovy v~Praze (dále 1.~LF) je pro centrální správu biologického materiálu používána aplikace BBM~\cite{1LF_BBM}. Systém definuje odběr tří typů vzorků: krev (plná krev, DNA, plazma), tkáň (tkáň, tkáň v~RNA lateru\footnote{Ukládanou látkou je RNA vyizolované z~plné krve. Later je činidlo zabraňující degradaci materiálu skladováním. Název later je odvozen od komerčně vyráběného produktu RNA later.}) a~moč. Aplikace umožňuje přidávat vzorky z~jednotlivých klinik pomocí formuláře. Data jsou přístupná přes webovou aplikaci.
Systém je napojen na NIS Medea\footnote{Dodavatelem je firma STAPRO s.r.o.~\cite{Medea}.} využívaný na VFN (Všeobecná fakultní nemocnice). 
Aplikace obsahuje následující moduly:  bioptický, tkáňový, sérový, plasmový, DNA, modul plné krve a~modul moči.
Typ materiálu je definován pomocí čtyř znaků (2 znaky typ + případně 2 znaková přípona \textit{-L} pro materiál uložený v~lateru). Počet vzorků uskladněných v~biobance se pohybuje v~řádu tisíců.

Interní číselník biologického materiálu používaný na 1.~LF se nachází v~příloze~\ref{tab:ciselnik-mat-Ilfuk}.

% ------------------------------------------------------------------------   
\subsection{FN Hradec Králové}
Ve Fakultní nemocnici v~Hradci Králové (dále FN HK) je používán starší NIS, neumožňující export požadovaných dat. Do budoucna se počítá s~tím, že při definování požadavků na nový systém budou zahrnuty i~požadavky na datový model plynoucí z~\ProjectName. Metadata o~vzorcích uložených v~hradecké biobance budou zatím zadávána manuálně prostřednictvím webového formuláře.

% ------------------------------------------------------------------------   
\subsection{FN Olomouc}
V~biobance Lékařské fakulty Univerzity Palackého (dále LF~UP) je používán SW od společnosti DS Soft\footnote{\url{http://www.dssoft.cz/}}. Systém eviduje seznam archivačních zařízení (Dewarovy nádoby apod.) s~požadovanými parametry (např. min. a~max. teplota), úložnou kapacitou a~aktuálním stavem zaplnění. Pro každé zařízení je definována vnitřní adresace vzorků.
V~biobance je evidováno řádově stovky vzorků, nicméně z~formálních důvodů bude umožněn export jen malé skupiny z~nich. Z~toho důvodu se v~tuto chvíli neplánuje implementace exportního modulu do používaného SW a~předpokládá se jejich ruční import.

Interní číselník biologického materiálu používaný v~Olomouci se nachází v~příloze~\ref{tab:ciselnik-mat-LFUP}.


% ------------------------------------------------------------------------   
% ------------------------------------------------------------------------   
\section{Popis exportu pacientských dat}

Faktorem ovlivňujícím podobu exportů byla míra (ne)shody ve způsobu evidence záznamů o~léčbě pacienta mezi nemocnicemi, kdy řada informací není zadávána strukturovaně, jednoznačně a~konzistentně. Snahou bylo vytvořit exportní schéma, reprezentující minimální množinu dat, která má význam pro výzkum a~přitom je realizovatelná s~minimem \uv{třecích ploch} mezi institucemi.

Jak je vidět v~kapitole~\ref{sec:instituce}, každá z~partnerských biobank ukládá trochu jinou množinu typů biologického materiálu. Z~laboratorního hlediska je mezi nimi velký rozdíl, ale z~informatického hlediska bylo možné je zobecnit na kategorie: tkáň, genom, sérum a~materiál se stanovenou diagnózou. Atributy, které se opakují, jsou popsány jen u~prvního typu, ve kterém jsou obsaženy.

Podstata této práce tkví v~informatické části projektu, takže popisu biologických detailů je věnován jen minimální prostor, pouze s~cílem, aby si čtenář bez biologických znalostí udělal základní představu dostačující pro pochopení datových modelů.
Data budou předávána ve formátu XML. Popisovaná schémata jsou součástí příloh.

% ------------------------------------------------------------------------   
\subsection{Pacient}
Veškerá data o~jednom pacientovi a~jeho vzorcích uložených v~biobance jsou uložena v~elementu \textit{patient}. Každý exportní soubor odpovídá veškerým datům souvisejícím s~jedinou léčenou osobou. Definice elementu \textit{patient} se nachází v~příloze~\ref{fig:export:data:patient}.

\begin{itemize}
		\polozka{Identifikátor biobanky} -- identifikátor deklarující, z~jaké instituce byl export vygenerován,

		\polozka{Identifikátor} -- identifikátor pacienta z~nemocnice, kde se léčil. Pro zaměstnance této nemocnice je tedy možné si tohoto pacienta zpětně dohledat ve svém NISu. Lokální identifikátor nesmí být přímým ani nepřímým nositelem žádného údaje pacienta, pomocí něhož by pacient mohl být identifikován z~vnějšku domovské instituce. Optimální je využití např. pořadových čísel z~NISu domovské instituce, jsou-li taková k~dispozici. Aby se zamezilo duplicitám mezi biobankami, je identifikátoru přidán prefix biobanky.
		
		\polozka{Informace o~narození} -- součástí exportu je rok a~měsíc narození pro stanovení přibližného stáří pacienta. Den narození (ani rodné číslo) předáváno není z~důvodu zachování anonymity.
		
		\polozka{Informovaný souhlas} -- explicitní souhlas pacienta s~účastí na výzkumu. Import nového pacienta bez souhlasu není vůbec zpracováván.
			
		\polozka{Pohlaví pacienta}
		
	\end{itemize}


% ------------------------------------------------------------------------   

\subsection{Tkáň}
V~kontextu práce projektu je tkání typicky vyoperovaný nádor nebo jeho bezprostřední okolí. Pro léčbu nádorových onemocnění je důležité charakterizovat, nakolik je tkáň podobná zdravé tkáni a~jaké známky chování vykazovala. K~tomu slouží klasifikace TNM a~MKN-O, popsané v~této kapitole. Ostatní parametry jsou organizačního charakteru (kolik čeho bylo kdy odebráno).
Definice elementu \textit{tissue} se nachází v~příloze~\ref{fig:export:data:tissue}.

\paragraphNewLine{Identifikace vzorku}
Vzorek je v~analyzovaných NISech identifikován identifikátorem složeným z~různých složek.
Příklad struktury indetifikátoru z~MOÚ a~1.~LF:
\begin{figure}[ht!]
\centering
\begin{BVerbatim}
MOÚ: 
rok + číslo odběru v rámci roku + typ vzorku 
+ označení alikvotu
1. LF: 
číslo pacienta + číslo odběru + typ vzorku 
+ označení alikvotu
\end{BVerbatim}
\end{figure}

Je vidět, že identifikátor není pouhý řetězec náhodných znaků, ale je složen z~jasně sémanticky definovaných částí. Jednotlivá pole identifikátoru lze z~exportu získat rozložením identifikátoru. Implementace by ale byla závislá na původu dat a~pro každou biobanku by musel existovat vlastní parser. Lepším postupem je tedy ponechání těchto polí v~samostatných atributech pro snížení složitosti implementace. 

V~exportním modelu je definován atribut \textit{sampleId}, který slouží pro následné dohledání vzorku v~NISu. Kvůli rozdílné struktuře identifikátorů je atribut v~exportu ponechán jako textový řetězec omezený jen délkou. Kromě \textit{sampleId} je možné definovat navíc ještě tzv. bioptickou žádanku -- tj. kombinaci roku a~pořadí odběru v~rámci roku.

Rozdíl mezi modelem NISů a~indexu \ProjectName je v~tom, že v~NISech je každý alikvot jedinečně identifikován, zatímco v~indexu jsou alikvoty považovány za vzájemně nerozlišitelné části vzorku. Část identifikátoru označující alikvotu není do \ProjectName importována.	
					
 \paragraphNewLine{Způsob odběru dat}					
	Položka způsob odběru dat popisuje, zda byl biologický materiál odebrán před operací, při operaci nebo po operaci. Pro případ, že tato informace není v~NISu ukládána ve strukturované podobě, je ponechaná možnost zadat \textit{unknown}.

\paragraphNewLine{Typ materiálu}
Typ materiálu představuje jemnější členění, o~jaký materiál se jedná, než definice samotného elementu (\textit{tissue}, \textit{genome}, \textit{serum}, \ldots). Typ materiálu nabývá hodnot z~proprietárních číselníků jednotlivých institucí. Hodnoty jsou v~indexu převáděny na interní číselník.
		
\paragraphNewLine{Datum odběru}
Datum a~čas, kdy došlo k~odběru materiálu. Pro odběr tkáně by bylo vhodnější použít termínu datum a~čas přerušení krevního zásobení, ale pro jednoduchost je používán termín datum odběru (element \textit{takingDate}) všude v~kontextu libovolného odběru biologického materiálu pacienta.
U~tkáně je navíc definováno pole \textit{freezeTime} tj. datum a~čas zamražení. To odpovídá době, kdy byly všechny alikvoty vzorku umístěny v~repozitáři biobanky.

\paragraphNewLine{Počet vzorků}
Počet alikvotů daného vzorku. Všechny alikvoty vzorku jsou považovány za identické a~vzájemně nerozlišitelné. 
Datový model obsahuje položku \textit{samplesNo}, která popisuje, kolik alikvotů daného vzorku je v~repozitáři biobanky uloženo. Související položkou je pak \textit{availableSamplesNo}, která definuje, kolik z~celkového počtu vzorků je možno nabídnout na výzkumné účely.

\begin{figure}[ht!] % Not floating 
\centering
\begin{BVerbatim}
Poč. nevydatelných = Celk. poč. - Poč. vydatelných
\end{BVerbatim}
\end{figure}
Dvě pole namísto jednoho jsou definována pro možnost rezervovat si vzorky pro interní potřeby instituce, např. servisní odběry popsané u~samožádanek. Nevydatelné vzorky slouží i~jako rezerva pro případná kontrolní měření, dojde-li k~pochybnosti o~výsledcích testů některého realizovaného výzkumu.

Situaci, kdy pracovníci biobank budou označovat všechny vzorky jako nevydatelné, aby si šetřili vlastní biologický materiál, musí předejít dohled koordinátora projektu. 

\paragraphNewLine{TNM a~pTNM}
TNM~\cite{TNM} je v~onkologii používaný systém klasifikace, popisující anatomický rozsah zhoubných nádoru. Podle fáze léčby se rozlišuje klasifikace klinická (cTNM) a patologická (pTNM). Před léčbou pacienta je stanovena klasifikace cTNM (občas označována jen jako TNM).

Po operačním vynětí tumoru (tj. nádoru) je stanoveno pTNM. Klasifikace pTNM vychází ze znalostí stavu před léčbou doplněných o~nálezy získané při chirurgickém výkonu a~o~patologické vyšetření. Tato klasifikace slouží k~odhadu dalšího vývoje a~konečných výsledků léčby. 

TNM se skládá ze tří částí:
\begin{itemize}
	\item T (Tumor) -- rozsah primárního nádoru. Popis může být rozšířen o~třetí znak upřesňující zařazení do podskupiny (např. $T1a$).
	\begin{compactitem}
		\item $TX$ -- nelze hodnotit
		\item $Tis$ -- karcinom \textit{in situ}\footnote{Laicky popsatelný jako neinvazivní nádor (nešíří se -- nevytváří metastáze).}
		\item $T0$ -- bez známek primárního nádoru
		\item $T1-n$ --zvětšující se velikost primárního nádoru
	\end{compactitem}
	
	\item N (Nodus) -- přítomnost/nepřítomnost a~případný rozsah metastázy\footnote{Metastáza je druhotné ložisko nádorových buněk, vzniklé odtržením z~primárního nádoru~\cite{metastaza}.} v~regionálních lymfatických uzlinách.
	\begin{compactitem}
		\item $NX$ -- nelze hodnotit
		\item $N0-n$ -- rozsah
	\end{compactitem}
	\item M (Metastasis) -- přítomnost/nepřítomnost vzdálených metastáz
\end{itemize}

\paragraphNewLine{Morfologie a~grading}
Patologickou morfologii lze velice laicky popsat jako charakteristiku nádoru a~jeho chování. Pro zápis se používá klasifikace MKN-O\footnote{Mezinárodní klasifikace nemocí pro onkologii. Vychází z~anglického ICD-O (International Classification of Disseases for Oncology). Aktuálně se používá třetí verze klasifikace.}~\cite{MKN-O}. Klasifikace je definovaná šesti číslicemi. První čtyři číslice vyjadřují mikroskopickou charakteristiku nádoru. Pátá, oddělena lomítkem, popisuje biologické chování nádoru v~těle (maligní\footnote{Maligní znamená \uv{škodlivý} a~v~češtině se používá ekvivalent zhoubný.}, benigní\footnote{Benigní znamená \uv{neškodný} a~v~češtině se používá ekvivalent nezhoubný.}, \textit{in situ}, nejisté). Šestá číslice popisuje tzv. \textit{grading}, vyjadřující v~jaké míře se nádor podobá tkáni, ze které vznikl (grading se uvádí pouze u~maligních nádorů). Pro případ, že v~některém NISu není morfologie evidována, je možné zadat pouze \textit{grading}.
Příklad morfologické klasifikace podle~\cite{MKN-O}:

\begin{figure}[ht!] % Not floating 
\centering
\begin{BVerbatim}
M - 8140/3 1
\end{BVerbatim}
\end{figure}

\subsection{Sérum}
Sérum je složka krve, kterou získáme odstředěním vysrážené plné krve. Definice elementu \textit{serum} se nachází v~příloze \ref{fig:export:data:serum}.

\subsection{Genomová krev}
Element \textit{genome} je pojmenován podle genomové krve -- tj. DNA vyizolované z~odebrané krve. Při uložení genomové krve, např. dle praxe na MOÚ, je do biobanky ukládána současně plná krev a~genomová DNA, obojí po jednom alikvotu.
Pro zmenšení složitosti exportních schémat je tento element použit k~popisu RNA, plné krve a~tkáně v~lateru.

Z~informatického pohledu by mohl být element \textit{genome} spojen s~elementem \textit{serum}. Důvodem pro vznik těchto podobných elementů bylo postupné zobecňování kategorií biologického materiálu na základě požadavků zapojených institucí. Rozhodnutí pro ponechání stávajícího rozdělení bylo spíš pragmatické, neboť do budoucna lze očekávat nárůst atributů a~specializace elementů. Definice elementu \textit{genome} se nachází v~příloze~\ref{fig:export:data:genome}.

\subsection{Materiál se stanovenou diagnózou}
Materiál se stanovenou diagnózou je obecná kategorie slučující typy materiálu, jejiž společnou charakteristikou je stanovená diagnóza. Do tohoto obecného typu spadá krev, moč a~jednotlivé extrahované složky krve.
Definice elementu \textit{diagnosisMaterial} se nachází v~příloze~\ref{fig:export:data:diagnosisMaterial}.

\paragraphNewLine{Diagnóza}
Pro popis diagnózy se používá klasifikace MKN-10\footnote{Mezinárodní klasifikace nemocí ve verzi 10. Vychází z~anglického originálu ICD-10 (International Classification of Disseases).}~\cite{MKN-10}, kterou lze popsat libovolnou známou chorobu.
Choroby jsou definované pomocí kódů o~délce tři až šest znaků. Páté a~šesté místo není pro index relevantní, neboť se týká specifikace pro index nezajímavých onemocnění a~zranění. 
Příklad diagnózy a~odpovídajícího zápisu MKN-10 klasifikace:

\begin{figure}[ht!]
\centering
\begin{BVerbatim}
C50.2 - Zhoubný novotvar prsu, horní vnitřní kvadrant prsu
\end{BVerbatim}
\end{figure}

% ------------------------------------------------------------------------   
\section{Popis exportu monitoringu biobank}

% ------------------------------------------------------------------------   
\subsection{Monitoring zaplnění biobank}\label{chapter:analysis:subsection:monitoring}
Základním prvkem LTS je Dewarova nádoba (v~modelu označena jako \textit{container}). Tu si lze představit jako válec s~kruhovým víkem, uvnitř něhož je několik sloupcových stojanů. Stojan (\textit{rack}) je konstrukce nesoucí ve sloupci několik krabiček (\textit{box}) se vzorky. Pro přístup k~jednomu vzorku je tedy potřeba otevřít Dewarovu nádobu, vyjmout příslušný stojan, z~něho vybrat správnou krabici a~v~té najít hledanou zkumavku (\textit{position}). Každý zmíněný prvek má své označení, takže spojením všech identifikátorů získáme jednoznačnou adresu vzorku. Jako příklad bude použit způsob uskladnění na MOÚ.

Vzorky uložené v~STS jsou na MOÚ uloženy v~mrazácích. Adresace je řešena na základě identifikace boxu bez dalšího hierarchického členění úložného prostoru.

Element \textit{biobank} obsahuje elementy \textit{container} nebo elementy \uv{samostatných} krabic odpovídajících primárně těm uloženým v~STS a~nebo sekundárně těm, které jsou adresované způsobem nekompatibilním s~modelem popisujícím hierarchii Dewarových nádob. Popis jednotlivých prvků elementu:

\begin{itemize}

\polozka{{Identifikátor}} -- pojmenování elementu. Identifikátor je použit pro adresaci, takže musí být unikátní v~kontextu nadřazeného elementu.

\polozka{{Umístění}} -- volitelný popis, kde se element nachází.

\polozka{{Kapacita}} -- maximální počet elementů, které se mohou uvnitř nacházet -- např. kolik stojanů pojme Dewarova nádoba.

\polozka{{Minimální teplota}} -- volitelný požadavek na mezní minimální teplotu pro daný prvek infrastruktury.

\polozka{{Maximální teplota}} -- volitelný požadavek na mezní maximální teplotu pro daný prvek infrastruktury.

\end{itemize}

% ------------------------------------------------------------------------   
\subsection{Monitoring teploty}
Sledování teploty v~úložné infrastruktuře bývá dle zjištěných informací delegováno na samostatný systém a~naměřené hodnoty nejsou běžně součástí NISů. 
V~případě MOÚ a~LF~UP je tato služba zajištěna monitorovacím systémem FALCON\footnote{Provozuje společnost KESA s.r.o. -- \url{www.kesa.cz}}. Ve FN HK a~na 1.~LF používají zařízení M355 a~SW CryoWatch dodávané k~zařízení od výrobce Taylor-Wharton. 

\paragraphNewLine{Integrace s~monitorovacím systémem FALCON}
Monitorovací systém FALCON umožňuje export dat do textového souboru o~následující struktuře.

\noindent{\textbf{První řádek}, tj. popis sloupců (pro přehlednější zobrazení výpis zalomen):} 
\begin{figure}[ht!]
\centering
\begin{BVerbatim}
číslo záznamu,číslo zákazníka,object_id,time_id,
objectstate,savereason,datum a čas, hodnota 0,
stav 0,hodnota 1,stav 1,hodnota 2,stav 2
\end{BVerbatim}
\end{figure}

\noindent{\textbf{Ostatní řádky} (data):}

\begin{figure}[ht!]
\centering
\begin{BVerbatim}
1,142,1,10115784,1,1,15.12.2013 10:14:20,5,1,4.1,1,0,1
2,142,1,10115802,1,1,15.12.2013 10:14:53,5,1,4.2,1,0,1
3,142,1,10116323,1,1,15.12.2013 10:28:54,5,1,4.5,1,0,1
...
\end{BVerbatim}
\end{figure}

Společnost interně eviduje mapování zákazníků a~sledovaných objektů. 
\textit{Time\_id} je interní sekvence reprezentující čas, kterou bude nutné při importu transformovat na klasickou reprezentaci času, používanou v~Javě, založenou na $ms$. \textit{Objectstate} reprezentuje stav celého objektu souhrnně za všechny čidla objektu. Pokud je jedno z~$n$ čidel ve stavu \textit{alarm}, je celý objekt ve stavu \textit{alarm}. Obdobně pokud je jediné čidlo ve stavu \textit{havárie}, pak je stav celého objektu \textit{havárie}, bez ohledu na čidla ve stavu \textit{alarm}. Parametr \textit{savereason} popisuje, jaká událost způsobila uložení hodnoty. Událostmi může být čas, alarm (souvisí s~překročením povolené teploty), havárie nebo fyzický přístup do Dewarovy nádoby (otevření víka je monitorováno). Parametr \textit{hodnota} a~\textit{stav} reprezentuje výstup z~konkrétního čidla měřeného objektu. Význam dat závisí na typu čidla a~definovaném číselníku. Číselníky umožňující další práci s~daty zatím nebyly poskytnuty.

Uvedený příklad popisuje dva teploměry a~indikaci otevření víka. Za referenční pro doložení kvality uložení se považuje výše položené čidlo, tj. to umístěné v~teplejším vzduchu dále od dusíkových par. 

\paragraphNewLine{Integrace s~ostatními biobankami}
SW CryoWatch umožňuje pouze manuální jednorázovou formu exportu dat do souboru formátu XLS nebo do textového souboru~\cite{M355CE}. Počítač je k~měřícímu přístroji připojen sériovým portem. Integrace tohoto systému není ve stávající fázi realizace projektu uvažována. 

% ------------------------------------------------------------------------   
\subsection{Předávání kalibračních protokolů}
Součástí požadavků na kvalitní uložení vzorků je také doložení, že teploměry jsou řádně zkalibrovány. Kalibrace se u~teploměrů vyžaduje minimálně jednou za dva roky. Díky nízké frekvenci je dostačujícím mechanismem manuální nahrání protokolu o~kalibraci do systému tak, aby byly k~dispozici koordinátorům projektu a~žadatelům o~vzorky.




% ------------------------------------------------------------------------   
% ------------------------------------------------------------------------      
% Návrh
% ------------------------------------------------------------------------   
\chapter{Návrh}\label{chapter:proposal}
Tato kapitola popisuje, jakým způsobem byly požadavky na systém převedeny do konkrétního návrhu SW architektury. Pro modelování systému byl použit modelovací jazyk UML.

% ------------------------------------------------------------------------   
% ------------------------------------------------------------------------   

\section{Architektura systému}
Nejrozšířenějším současným paradigma vývoje webových aplikací je tzv. třívrstvá architektura. Princip spočívá v~oddělení logiky aplikace (aplikační vrtva) od způsobu uložení (perzistenční vrstva) a~prezentace dat (prezentační vrstva). Každou vrstu si lze představit jako server, který poskytuje své služby vyšší vrstvě. Prezentační (tj. nejvyšší) poskytuje služby uživateli a~perzistenční (nejnižší) využívá služeb datového úložiště. Každá úroveň má vymezenou kompetenci a~odpovědnost.
Některé z~výhod třívrstvé architektury:
\begin{itemize}
	\item Omezuje množství závislostí a~je v~ní jasně definovaný směr závislostí (vyšší úroveň využívá nižší). To zlepšuje přehlednost a~udržovatelnost kódu.
	\item Lze snadno nahradit implementaci některé z~vrstev.
	\item Snazší delegování práce mezi programátory a~usnadnění komunikace díky společnému slovníku.
	\item Frameworky jsou vytvářeny tak, aby naplňovaly požadavky jednotlivých vrstev. Díky obdobně definovaným kompetencím ve všech třívrstvých aplikacích je snazší frameworky integrovat.
\end{itemize}

\paragraphNewLine{Perzistenční vrstva}
Perzistenční vstva má na starosti správu dat, jejich uložení, získání i~mazání (tzv. CRUD). Rozhraní perzistenční vrstvy má v~obecném pojetí odstínit aplikační logiku od implementačních detailů uložení dat. Ostatní vrstvy by měly být nezávislé na tom, jaká implementace databáze je využita nebo kde se data nachází (lokální DB, vzdálená DB, uložení v~XML, atd.).
Perzistenční vrstva se skládá z~tzv. DAO tříd\footnote{Podle návrhového vzoru Data Access Object používaného pro Javu EE.} a~každá obsahuje veškerou logiku pro práci s~konkrétní entitou systému.

\paragraphNewLine{Aplikační vrstva} 
Na aplikační (též často nazývané jako servisní) vrstvě je vykonávána veškerá logika aplikace, tj. to co je po systému požadováno za funkcionalitu. Vrstva přebírá požadavky z~prezentační vrstvy a~zajišťuje vykonání příslušných požadavků na straně serveru. 

\paragraphNewLine{Prezentační vrstva} 
Slouží ke zpracování požadavků klientů a~ke generování odpovědí zobrazovaným v~prohlížeči klienta. Prezentační vrstva má na starost validaci zadávaných dat a~komunikaci s~uživatelem prostřednictvím chybových či potvrzujících hlášení o~výsledku operací vykonaných aplikační vrstvou. 

\subsection{Společné vlastnosti entit}
Všechny entity ukládané do databáze jsou identifikované pomocí celého čísla generovaného databází. V~některých situacích by bylo možné se tomu vyhnout např. použitím unikátního řetězce znaků (biobanka, pacient, \ldots), ale použití stejného typu identifikátoru v~celé aplikaci výrazně zjednodušuje složitost rozhraní a~následné implementace. Druhým argumentem pro celočíselný identifikátor je snazší manipulace na prezentační vrstvě. Identifikátor je na mnoha místech parametrem URL. Pro případ identifikátoru tvořeného řetězcem znaků by vznikalo \uv{nepěkné} URL a~mohlo by docházet k~problémům způsobených tím, že v~URL jsou dovolené pouze ASCII znaky.

% ------------------------------------------------------------------------   
% ------------------------------------------------------------------------   
\section{Uživatel}
Diagram tříd~\ref{fig:index:uml:class:user} popisuje entitu \textit{User}, tj. uživatele systému, a~bezprostřední vazby s~ní související. Většina hodnot atributů třídy je získána při prvním přihlášení uživatele z~atributů hlavičky HTTP requestu zaslané poskytovatelem identit (v~rámci federace eduId). Při přístupu k~produkčnímu serveru proběhně autentizace vůči \uv{domovské} instituci uživatele aniž by aplikace manipulovala s~heslem uživatele. Atribut \textit{password}, který je součástí třídy \textit{User} slouží pouze pro vývoj a~testování. 
Při každém dalším přihlášení jsou údaje zaslané z~eduId porovnávány s~těmi v~databázi a~pokud došlo ke změně, jsou ty v~databázi přepsány novými. Jediným uživatelsky změnitelným údajem je e-mail, aby si mohl uživatel sám zvolit, jakou kontaktní e-mailovou adresu preferuje.

Třída \textit{UserSetting} poskytuje možnost uložit individuální nastavení aplikace.
Pro změny chování celé aplikace je možné definovat nové instance třídy \textit{GlobalSetting}. Nastavení není vyžadováno v~požadavcích na systém a~mohlo by se do jisté míry řešit použitím konstanty v~implementaci. Do budoucna lze očekávat rozšíření uživatelsky nebo systémové nastavitelných atributů a~je vhodné s~tím dopředu počítat.

\begin{figure}[h!]
\centering
	\includegraphics[width=\textwidth]{UserView}
\caption{UML class diagram tříd souvisejících s~třídou \textit{User}.}
\label{fig:index:uml:class:user}
\end{figure}

\subsection{Systémové role}
Systémové role jsou nástrojem řízení přístupu v~aplikaci. Každý uživatel může mít přiděleno několik systémových rolí, podle kterých mu jsou zpřístupněny různé části webového rozhraní. Systémové role jsou definovány enumerační třídou \textit{SystemRole}. 
U~entit \textit{Project} a~\textit{Biobank} jsou nad rámec systémových rolí definovany ještě oprávnění (\textit{Permission}), která stanovují co může uživatel s~konkrétní instancí vykonávat za operace.

\paragraphNewLine{User}
\textit{User}\footnote{Pro lepší srozumitelnost textu: uživatel = osoba využívající systém, zatímco \textit{user} = systémová role.} je implicitní role každého uživatele oprávněného přistupovat k~indexu. Při autentizaci uživatele je na základě parametrů získaných z~eduId stanoveno, zda může být dotyčnému uživateli umožněno vstoupit do systému. Každý kdo je vpuštěn má roli \textit{user}. Existence uživatele bez role \textit{user} uvnitř systému je považována za chybu.
Uživatel s~touto rolí je opravněn:
\begin{compactitem}
	\item přistupovat ke svým vlastním údajům a~k~osobnímu nastavení,
	\item podávat rezervace na vzorky,
	\item zakládat nové projekty,
	\item zobrazit kontakty na vývojáře a~administrátory systému,
	\item zobrazit všechny biobanky registrované v~systému.
\end{compactitem}

\paragraphNewLine{Správce biobanky}
Uživateli s~oprávněním alespoň k~jedné biobance je přidělena systémová role \textit{správce biobanky}. Role samotná upravuje pouze položky v~menu. Autorizace pro provádění konkrétních operací je stanovena na základě oprávnění ke konkrétní instanci biobanky. Oprávnění jsou popsána v~kapitole \ref{chapter:proposal:subsection:biobankPermission}.  

\paragraphNewLine{Člen projektového týmu} 
Každý uživatel s~oprávněním alespoň k~jednomu projektu má přidělenu systémovou roli \textit{člen projektového týmu}. Role sama o~sobě neopravňuje uživatele k~ničemu popsanému v~kapitole \ref{chapter:analysis:subsection:authorization}. Role je použita při vyhodnocování, zda uživatel může editovat projekt. Uživatel bez role \textit{člen projektového týmu} nemá přístup ke správě projektů. Pro uživatele disponující touto rolí je ověřováno, zda pro konkrétní projekt má nějaké oprávnění. Oprávnění jsou popsána v~kapitole \ref{chapter:proposal:subsection:projectPermission}.  

\paragraphNewLine{Člen projektového týmu se schváleným projektem}
Systémová role \textit{člena projektového týmu s~potvrzenym projektem} reprezentuje statut oprávněné osoby. Tato systémová role je přidělena všem uživatelům s~oprávněním alespoň k~jednomu akceptovanému projektu. Role \textit{člen projektového týmu se schváleným projektem} je uživateli přidána při potvrzení projektu. Role je odebrána při smazání nebo dokončení posledního schváleného projektu uživatele.
Uživatel s~touto rolí je opravněn:
\begin{compactitem}
	\item zobrazit monitoring teploty ve všech biobankách,
	\item vytvářet žádosti o~vzorky.
\end{compactitem}

\paragraphNewLine{Vývojář}
Systémová role \textit{vývojář} je určena pro uživatele systému, kteří se podílí na jeho vývoji.
Uživatel s~touto rolí je opravněn:
\begin{compactitem}
	\item založit biobanku,
	\item smazat biobanku,
	\item zobrazit veškeré informace zobrazované na webu (bez možnosti editace),
	\item přiřadit nebo odebrat uživateli roli vývojáře nebo administrátora,
	\item zobrazit archiv událostí.
\end{compactitem}

\paragraphNewLine{Administrátor}
Systémová role \textit{administrátor} přísluší koordinátorům projektu \ProjectName.
Uživateli s~touto rolí je dovoleno:
\begin{compactitem}
	\item založit biobanku, 
	\item smazat biobanku,
	\item zobrazit veškeré informace zobrazované na webu (bez možnosti editace),
	\item přístup ke globálnímu nastavení systému,
	\item přiřadit nebo odebrat uživateli roli vývojáře nebo administrátora,
	\item zobrazit archiv událostí.
\end{compactitem}

% ------------------------------------------------------------------------   
% ------------------------------------------------------------------------   
\section{Projekt}
Třída \textit{Project} reprezentuje záznam o~existujícím podaném projektu. Většina ukládaných atributů vychází z~formálních náležitostí souvisejících s~existujícím projektovým záměrem:
\begin{compactitem}
	\item \textit{name} -- název projektu,
	\item \textit{fundingOrganization} -- kdo projekt financuje,	
	\item \textit{approvedBy} -- kým byl projekt schválen. Schvalováním je v~tomto kontextu myšlen proces mimo aplikaci -- tj. nikoli formální kontrola administrátory \ProjectName,
	\item \textit{approvalStorage} -- kde je fyzický souhlas uložen,
	\item \textit{principalInvestigator} -- hlavní řešitel projektu. Atribut je ukládán jako textový řetězec pro případ, že by dotyčně osobě nebyl umožněn přístup do systému (např. zaměstnanec instituce nezačleněné do federace eduId),
	\item \textit{homeInstitution} -- instituce, která projekt zaštiťuje,
	\item \textit{approvalDate} -- datum schválení projektového záměru,
	\item \textit{annotation} -- textový popis projektového záměru.
\end{compactitem}
Všechny jmenované atributy kromě anotace jsou považované za konstantní v~průběhu životního cyklu projektu. 

\begin{figure}[h!]
\centering
	\includegraphics[width=\textwidth]{ProjectView}
\caption{UML class diagram tříd souvisejících s~projektem.}
\label{fig:index:uml:class:project}
\end{figure}

% ------------------------------------------------------------------------   
\subsection{Příloha projektu}
Při nahrání projektu je nutné nahrát podepsaný formulář MTA. To je jediný nutný dokument, který musí přílohy projektu obsahovat. 
Model aplikace nicméně umožňuje, aby bylo nahráno libovolné množství příloh k~projektu. Pro každou přílohu je možné definovat, jaký je její význam v~rámci projektu pro usnadnění orientace mezi soubory. U~kategorizace příloh se nepočítá s~velkými změnami, takže byl pro jejich definování použit výčtový typ.

% ------------------------------------------------------------------------   
\subsection{Životní cyklus projektu}
Po založení je projekt označen jako \textit{nový}. Po kontrole formálních náležitostí administrátorem může být \textit{akceptován} nebo \textit{zamítnut}. V~kontextu schváleného projektu je již možné žádat o~vzorky. Projekt s~alespoň jednou žádostí je považován za \textit{zahájený}. Na konci realizace lze označit projekt za \textit{dokončený}. \textit{Dokončený} projekt v~systému i~nadále zůstává pro archivaci, ale je již pouze pro čtení a~nelze v~jeho kontextu žádat o~vzorky. 
V~aplikaci je definována i~možnost zrušit schválený projekt (v~diagramu značeno přerušovanou čarou). Tato funkce není ve stávající verzi zpřístupněna na uživatelském webovém rozhraní. Životní cyklus projektu je znázorněn na diagramu~\ref{fig:implementace:projekt:cyklus}.

\begin{figure}[htbp]
\centering
\begin{tikzpicture}[->,>=stealth',shorten >=1pt,auto,node distance=45mm,
  thick,main node/.style={ellipse,draw}]

  \node[main node] (novy) {Nový};
  \node[main node] (akceptovany) [right of=novy] {Akceptovaný};
  \node[main node] (zamitnuty) [below = 20mm of akceptovany] {Zamítnutý};
  \node[main node] (zapocaty) [right of= akceptovany] {Započatý};
  \node[main node,  dashed] (zruseny) [above = 20mm of zapocaty] {Zrušený};
  \node[main node] (dokonceny) [below = 20mm of zapocaty] {Dokončený};

  \path[every node/.style={font=\sffamily\small}]
    (novy) 
        edge [right, palatinatepurple] node[sloped, above] {Schválit} (akceptovany)
        edge [right, palatinatepurple] node[sloped, above] {Zamítnout} (zamitnuty)
    (akceptovany)
				edge [right, cyan] node[sloped, above] {Žádost} (zapocaty)
        edge [right, cyan] node[sloped, above] {Dokončit} (dokonceny)
        edge [right, dashed, palatinatepurple] node[sloped, above] {Zrušit} (zruseny)
    (zamitnuty)
	  (zapocaty)
				edge [right, cyan] node[right] {Dokončit} (dokonceny)
        edge [right, dashed, palatinatepurple] node[right] {Zrušit} (zruseny)
    (zruseny)
    (dokonceny);

\end{tikzpicture}
\caption{Životní cyklus projektu. Iniciátor události odlišen barvou šipky: {\color{palatinatepurple}správce biobanky}, {\color{cyan}člen projektového
týmu}.}
\label{fig:implementace:projekt:cyklus}
\end{figure}

% ------------------------------------------------------------------------   
\subsection{Oprávnění k~projektu}\label{chapter:proposal:subsection:projectPermission}
V~průběhu realizace se ukázalo, že je žádoucí, aby kardinalita mezi projektem a~uživatelem nebyla $n:1$, ale $n:m$. Oprávnění k~úpravě projektu ale nemůže být vázané na systémovou roli, protože pak by při špatném rozvržení prezentační vrstvy hrozilo, že bude umožněna manipulace s~cizím projektem. Z~hlediska spolupráce více lidí na projektu je také vhodné umožnit delegování odpovědnosti. Tyto argumenty vedly k~potřebě rozlišovat úroveň oprávnění přístupu zvlášť ke každému projektu.
Mezi jednotlivými oprávněními je vztah inkluze, tj. vyšší zahrnuje vše z~nižšího. Oprávnění jsou definována v~třídě \textit{Permission}.

\paragraphNewLine{Návštěvník (Visitor)} 
Uživateli s~oprávněním \textit{visitor} je umožněno zobrazit všechny stránky související s~projektem, ke kterému se oprávnění pojí -- tj. detail projektu, projektový tým, přílohy a~další. Není mu umožněno data jakkoli měnit.

\paragraphNewLine{Výkonný pracovník (Executor)} 
Oprávnění \textit{executor} umožňuje uživateli vykonávat následující operace:
\begin{compactitem}
	\item označit projekt jako dokončený,
	\item pracovat s~žádostmi o~vzorky (vytvoření, schvalování,\ldots),
	\item manipulace s~přílohami projektu.
\end{compactitem}

\paragraphNewLine{Pracovník s~možností editace (Editor)}
Uživateli s~oprávněním \textit{editor} je umožněno měnit editovatelné položky související s~projektem.

\paragraphNewLine{Manager}
Nejvyšší oprávnění umožňující manipulovat se složením týmu a~s~oprávněními jednotlivých osob v~týmu.

% ------------------------------------------------------------------------   
\subsection{Životní cyklus žádosti a~rezervace vzorků}
Uživatel s~projektovým záměrem, ale bez splněných formálních náležitostí, může podat žádost o~rezervaci vzorků. Ta je \textit{schválena} v~situaci, kdy banka má kapacitu na to, aby žádosti vyhověla (tj. mají dostatek vzorků). V~opačném případě je rezervace \textit{zamítnuta}. Ke schválené rezervaci je možné přiřadit konkrétní počty alikvotů pro jednotlivé vzorků. Když je seznam vzorků kompletní, je žádost \textit{uzavřena}. Od chvíle uzavření běží lhůta platnosti rezervace. Dobu rezervace může nastavovat \textit{administrátor} pomocí globálního nastavení aplikace, které poskytuje třída \textit{GlobalSetting}. Lhůta je definována shodně pro celý systém. Výchozí hodnota je stanovena konstantou ve třídě \textit{Constant}. Když tato lhůta vyprší, je žádost označena jako \textit{expirovaná} a~alokované vzorky jsou uvolněny.

Podle kapitoly~\ref{chapter:analysis:section:workflow} nemohou být uživateli vzorky vydány pouze na základě rezervace. Uživatel musí pro získání rezervovaných vzorků nejprve zadat do systému projekt, pro který vytvářel rezervaci. Projekt projde běžnou kontrolou na splnění formálních náležitostí, a~když je administrátory akceptován, lze k~němu přiřadit rezervaci. Rezervace se tím změní na žádost (\textit{SampleRequest}), na základě které už mohou být vzorky poskytnuty.

Životní cyklus rezervace je znázorněn v~diagramu~\ref{fig:implementace:rezervace:cyklus}, cyklus žádosti je v~diagramu~\ref{fig:implementace:zadost:cyklus}.

\begin{figure}[htbp]
\centering
\begin{tikzpicture}[->,>=stealth',shorten >=1pt,auto,node distance=45mm,
  thick,main node/.style={ellipse,draw}]

  \node[main node] (novy) {Nová};
  \node[main node] (schvalen) [right of=novy] {Schválena};
  \node[main node] (zamitnut) [below = 20mm of schvalen] {Zamítnuta};
  \node[main node] (uzavren) [right of=schvalen] {Uzavřena};
  \node[main node] (expirovan) [above = 20mm of uzavren] {Expirovala};
	\node[main node] (zadost) [below = 20mm of uzavren] {Vytvořena žádost};
    
  \path[every node/.style={font=\sffamily\small}]
    (novy)
    	edge [right, palatinatepurple] node[sloped, above] {Schválit} (schvalen)
      edge [right, palatinatepurple] node[sloped, above] {Zamítnout} (zamitnut)
    (schvalen)
    	edge [right, palatinatepurple] node[sloped, above] {Zkompletovat} (uzavren)
    (zamitnut)
    (uzavren)
      edge [right] node[right] {Expirace} (expirovan)
			edge [right] node[right, cyan] {Přiřadit projektu} (zadost);
\end{tikzpicture}
\caption{Životní cyklus rezervace. Iniciátor události odlišen barvou šipky: {\color{palatinatepurple}správce biobanky}, {\color{cyan}uživatel}.}
\label{fig:implementace:rezervace:cyklus}
\end{figure}

Počátek životního cyklu žádosti o~vzorky je totožný s~počátkem životního cyklu rezervace. V~situaci, kdy je žádost uzavřena pracovníkem biobanky, je umožněno, aby žadatel sadu potvrdil, nebo ji vrátil zpět k~doplnění. Tato volba se uplatní v~situaci, kdy vzorky nesplňují kladené požadavky. Žádost se tím vrátí do stavu \textit{schválena}, aby bylo opět možné editovat sadu vzorků.
Může také nastat situace, že o~stejné vzorky bylo požádáno ve více biobankách a~více žádostí bylo vyřízeno kladně, takže je pro projekt alokováno více zdrojů, než je nutné. V~této situaci může správce projektu žádost smazat a~tím alokované vzorky uvolnit.
Po schválení vybrané sady žadateli jsou tyto vzorky fyzicky připraveny k~odebrání. Žádost s~fyzicky předanými vzorky je označena jako \textit{doručena}.

\begin{figure}[htbp]
\centering
\begin{tikzpicture}[->,>=stealth',shorten >=1pt,auto,node distance=45mm,
  thick,main node/.style={ellipse,draw}]

  \node[main node] (novy) {Nová};
  \node[main node] (schvalen) [right of=novy] {Schválena};
  \node[main node] (zamitnut) [below = 20mm of schvalen] {Zamítnuta};
  \node[main node] (uzavren) [right of = schvalen] {Uzavřena};
  \node[main node] (potvrzen) [below = 20mm of uzavren] {Potvrzena};
  \node[main node] (vydan) [below = 20mm of potvrzen] {Vybavena};
    
  \path[every node/.style={font=\sffamily\small}]
    (novy)
    	edge [right, palatinatepurple] node[sloped, above] {Schválit} (schvalen)
        edge [right, palatinatepurple] node[sloped, above] {Zamítnout} (zamitnut)
    (schvalen)
    	edge [right, palatinatepurple] node[sloped, above] {Zkompletovat} (uzavren)
    (zamitnut)
    	
    (uzavren)
    	edge [right, cyan] node[right] {Vyhovuje} (potvrzen)
        edge [bend left, cyan] node[sloped, above] {Nevyhovuje} (schvalen)
    (potvrzen) 
    	edge [right, palatinatepurple] node[right] {Předat} (vydan)
    ;

\end{tikzpicture}
\caption{Životní cyklus žádosti. Iniciátor události odlišen barvou šipky: {\color{palatinatepurple}správce biobanky}, {\color{cyan}člen projektového
týmu}.}
\label{fig:implementace:zadost:cyklus}
\end{figure}

% ------------------------------------------------------------------------   
% ------------------------------------------------------------------------   
\section{Biobanka}
Entita reprezentující instituci zapojenou do projektu \ProjectName, která poskytuje biologický materiál ostatním. Atribut \textit{abbreviation} slouží pro kontrolu při parsování importů, že zpracovávaný XML soubor skutečně patří ke správné biobance a~nedošlo k~chybě při kopírování (např. záměna cílové cesty).

\begin{figure}[h!]
\centering
	\includegraphics[width=\textwidth]{BiobankView}
\caption{UML class diagram tříd souvisejících s~biobankou.}
\label{fig:index:uml:class:biobank}
\end{figure}

% ------------------------------------------------------------------------   
\subsection{Infrastruktura biobanky}
Diagram~\ref{fig:index:uml:class:infrastructure} popisuje, jakým způsobem je v~aplikaci uložena struktura repozitáře biobanky. Struktura vychází z~analýzy popsané v~kapitole~\ref{chapter:analysis:subsection:monitoring}.

\begin{figure}[h!]
\centering
	\includegraphics[width=\textwidth]{InfrastructureView}
\caption{UML class diagram popisující infrastrukturu biobanky.}
\label{fig:index:uml:class:infrastructure}
\end{figure}

% ------------------------------------------------------------------------   
\subsection{Monitoring biobanky}
Kontejner může mít přiřazeno $0-n$ měření. Každé měření definuje pomocí výčtového typu konkrétní měřenou fyzikální veličinu a~použité jednotky. Touto strukturou lze tedy modelovat různé počty teplotních i~jiných čidel, přiřazených ke kontejneru. 
Doba ponechání změřených dat v~databázi bude stanovena až na základě praktických zkušeností s~využíváním systému (faktické množství generovaných dat proti požadavkům na zpětné doložení nakládání s~daty).

\begin{figure}[h!]
\centering
	\includegraphics[width=\textwidth]{MonitoringView}
\caption{UML class diagram popisující sledování fyzikálních parametrů v~prvcích infrastruktury biobanky.}
\label{fig:index:uml:class:monitoring}
\end{figure}


% ------------------------------------------------------------------------   
\subsection{Kalibrační protokoly}
Pro ověření, že data z~monitoringu biobanky jsou relevantní k~posouzení kvality uložení, je třeba minimálně jednou za dva roky měřící přístroje zkalibrovat. Protokoly o~kalibraci je možné uložit jako dokumenty vztahující se k~biobance.

% ------------------------------------------------------------------------ 

\subsection{Oprávnění k~biobance}\label{chapter:proposal:subsection:biobankPermission}
Stejná motivace jaká je zmíněna u~projektu vedla k~zavedení úrovní oprávnění i~pro biobanku. Oprávněními je možné rozlišovat odpovědnost jednotlivých správců biobanky. Model umoňuje, aby uživatel byl správcem více biobank, byť to v~tuto chvíli není pravděpodobným scénářem.
Oprávnění pro biobanku jsou definována následovně.

\paragraphNewLine{Návštěvník (Visitor)} 
Obdobně jako u~projektu, je i~u~biobanky oprávnění visitor omezeno výhradně na zobrazení dat bez možnosti jejich úpravy. 

\paragraphNewLine{Výkonný pracovník (Executor)}
Uživatel s~oprávněním \textit{executor} je oprávněn v~kontextu biobanky vykovánavat následující operace:
\begin{compactitem}
	\item schvalovat projekty,
	\item manipulovat s~žádostmi o~vzorky.
\end{compactitem}

\paragraphNewLine{Pracovník s~možností editace (Editor)} 
Uživatel s~oprávněním \textit{editor} je oprávněn v~kontextu biobanky vykovánavat následující operace:
\begin{compactitem}
	\item měnit editovatelné položky biobanky,
	\item manuální vytváření infrastruktury biobanky (kontejnery, stojany, ...),
	\item manuální vytváření pacientů a~vzorků.
\end{compactitem}

\paragraphNewLine{Manager}
Nejvyšší oprávnění umožňující manipulovat se složením týmu a~s~oprávněním jednotlivých osob.

% ------------------------------------------------------------------------   
% ------------------------------------------------------------------------   
\section{Popis uložených vzorků}
Class diagram~\ref{fig:index:uml:class:sample} vychází ze struktury exportů popsaných v~předchozí kapitole.

\begin{figure}[h!]
\centering
	\includegraphics[width=\textwidth]{SampleView}
\caption{UML class diagram popisující ukládaná metadata vzorků.}
\label{fig:index:uml:class:sample}
\end{figure}

% ------------------------------------------------------------------------   
% ------------------------------------------------------------------------      
% Implementace
% ------------------------------------------------------------------------   
\chapter{Implementace}\label{chapter:implementation}
Aplikace je provozována na serveru Metacentra\footnote{\url{http://www.metacentrum.cz/}}, na adrese cloud19.cerit-sc.cz. Druhý server, přístupný na adrese cloud20.cerit-sc.cz, slouží pro testovací účely. V~obou případech se jedná o~linuxové stroje s~operačním systémem Debian Squeeze verze 6.0. Stroje jsou vybaveny procesory Intel Xeon E312xx Sandy Bridge a~disponují 6GB operační paměti. Pro nasazení aplikace byly zaregistrovány domény \url{www.bbmri.cz} a~\url{index.bbmri.cz}, které odkazují na produkční server \url{cloud19.cerit-sc.cz}. Domény jsou spravovány informatickým oddělením MOÚ.

Dle požadavků bylo nutné zajistit šifrování komunikace se serverem. Proto byl pro uvedené domény vystaven certifikát, používaný pro HTTPS. 
Doména \url{www.bbmri.cz} má čistě informativní charakter, zatímco doména \url{https://index.bbmri.cz} vede do vlastní aplikace.
Pro vývoj aplikace byl používán repozitář github\footnote{\url{https://github.com/Ondrej-vojtisek/bbmri}}. 

Cílem následující kapitoly je vysvělit argumenty pro volbu použitých technologií a~vysvětlit některé nestandardní implementační detaily.

% ------------------------------------------------------------------------   
% ------------------------------------------------------------------------    
\section{Použité technologie}
Aplikace je implementována v~Javě verze 7, nicméně nevyužívá žádných syntaktických \uv{novinek} Javy 7 a~kód je bez úprav kompilovatelný a~spustitelný i~pro Javu verze 6. 

% ------------------------------------------------------------------------   
\subsection{Maven}
Pro sestavení aplikace byl používán nástroj Maven\footnote{\url{http://maven.apache.org/}}. Programátor v~souboru pom.xml nadefinuje, které knihovny a~programovací nástroje aplikace používá, včetně specifikace verze těchto nástrojů. Maven tyto závislosti stáhne a~aplikaci sestaví. Pro vývoj představuje použití Mavenu usnadnění, protože vývojář nemusí řešit tranzitivní závislosti jednotlivých knihoven a~má lepší přehled nad verzemi použitých knihoven. 

% ------------------------------------------------------------------------   
\subsection{Hibernate}
Pro ukládání dat a~pro přístup k~nim byl použit ORM (Object-Relational Mapping\footnote{Způsob konverze dat mezi relační databází a~objektem objektově orientovaného jazyka.}) JPA (Java Persistence API). JPA je rozhraní (API), které definuje, jakým způsobem jsou objekty mapovány na tabulky relační databáze. Vazby mezi entitami aplikace jsou definovány pomocí anotací ve zdrojovém kódu. Parametry připojení k~databázi jsou JPA poskytnuty konfiguračními soubory viz~\ref{chapter:implementation:section:configuration}. JPA není závislé na konkrétní implementaci databáze.

Použitou implementací JPA v~aplikaci byl Hibernace ve verzi 3.6.10. Alternativou k~Hibernate je např. OpenJPA\footnote{\url{https://openjpa.apache.org/}}. Argumentem pro volbu Hibernate byl explicitní požadavek v~zadání práce, stanovený s~cílem sjednotit používané technologie v~projektech realizovaných na Ústavu výpočetní techniky MU (resp. v~centru CERIT-SC). 
Hibernate za vývojáře zajišťuje vytvoření databáze podle datového modelu (popsaného anotacemi entit aplikace). Hibernate zvládne i~průběžně měnit strukturu databáze podle aktuálních změn anotací. Vývojář musí mít pouze na paměti, že po úpravách mohou v~databázi zůstat nekonzistentní objekty.
Pro dotazování nad databází se používá JPQL (Java Persistence Query Language) namísto SQL běžně používaného v~relačních databázích. JPQL umožňuje pokládat dotazy na základě znalosti entit a~bez znalosti jejich skutečného uložení v~databázi (např. kladení dotazů bez znalosti pojmenování tabulek).

% ------------------------------------------------------------------------   
\subsection{Spring}
Jako základ pro implementaci aplikační vrstvy aplikace byl použit aplikační rámec Spring\footnote{\url{http://spring.io/}}, který umožňuje automatické dynamické vytváření potřebných objektů za běhu aplikace. Tento mechanismus se nazývá \textit{dependency injection} a~pro programátora představuje usnadnění, neboť programátor nemusí dohlížet na incializaci kontrolovaných objektů. Objekty, jejichž závislost má spravovat Spring, jsou v~kódu označeny anotací \textit{@Autowired}. Jedná se o~všechny třídy implementující rozhraní DAO a~Servisní (resp. aplikační) vrstvy.
Alternativou pro Spring je aplikační framework EJB\footnote{\url{http://www.oracle.com/technetwork/java/javaee/ejb/index.html}}.

% ------------------------------------------------------------------------   
\subsection{Stripes}
Stripes je MVC (Model-View-Controller) aplikační rámec pro prezentační vrstvu Java EE aplikací. Hlavní myšlenkou autorů frameworku byla snaha minimalizovat nutnou konfiguraci pomocí XML souborů a~umožnit konfiguraci pomocí anotací zavedených v~Javě 5. Stripes řeší vše nutné pro webovou vrstvu aplikace, tedy například validaci vstupů, lokalizaci obsahu, mechanismus informování uživatele zprávami, atd. Stripes mimo jiné obsahuje nástroje pro tvorbu šablon JSP stránek, které výrazně napomáhají snížit duplicitu kódu.
Využití frameworku bylo doporučeno na základě zvyklostí s~vývojem Java EE aplikací na ÚVT MU.

\paragraph*{Stripesstuff} -- Použitá sada rozšíření k~webovému rámci Stripes, usnadňující mimo jiné implementaci zabezpečení a~lokalizace webové aplikace. 

% ------------------------------------------------------------------------   
\subsection{Styl webu a~JavaScript}
Pro vizuální vzhled aplikace byl vybrán webový rámec bootstrap\footnote{\url{http://getbootstrap.com/}} ve verzi 2.3.2. Bootstrap je kolekce HTML kódu, CSS stylů a~u~dynamičtějších prvků webu (např. panel pro výběr data nebo výběr a~nahrání souboru z~disku) i~souvisejících jQuery knihoven. Společně tak vytváří pestrou sadu webových komponent s~jednotnou vizuální podobou, umožňující rychlé nasazení prototypu s~prezentovatelným designem. 

JQuery bylo použito kromě komponent Bootstrapu také pro zobrazení dat z~monitoringu fyzikálních parametrů biobank v~grafu. Pro tuto funkci byla použita knihovna Flot\footnote{\url{www.flotcharts.org}}. 


% ------------------------------------------------------------------------   
% ------------------------------------------------------------------------   
\section{Práce s~daty}

% ------------------------------------------------------------------------   
\subsection{Organizace uložení dat}\label{chapter:implementation:subsection:organizationOfDataStorage}
Aplikace manipuluje s~množstvím různorodých dat, která se budou v~systému nacházet různě dlouhou dobu a~která do systému vstupují různými způsoby. Pro organizaci souborů aplikace slouží složka definovaná parametrem \textit{StoragePath} v~konfiguračním souboru \textit{my.properties} (výchozí hodnota je \textit{/home/bbmri\_data}). Tento adresář slouží k~uložení veškerých dat souvisejících s~aplikací.

Projektové soubory (\textit{ProjectAttachment} v~UML diagramech) jsou uloženy uvnitř složky \textit{project\_files/id}, kde \textit{id} je identifikátor projektu. 

Soubory související s~biobankou jsou uloženy ve složce \textit{biobank\_files/id}, kde \textit{id} je identifikátor biobanky. Adresář obsahuje následující podsložky: 
\begin{compactitem}
	\item \textit{patient\_data} -- pacientská data a~metadata vzorků,
	\item \textit{monitoring\_data} -- data popisující zaplnění biobanky,
	\item \textit{temperature\_data} -- výstupy z~teploměrů v infrastruktuře,
	\item \textit{calibration\_data} -- zaslané kalibrační protokoly,
	\item \textit{patient\_data\_archive} -- archiv pacientských dat,
	\item \textit{monitoring\_data\_archive} -- archiv zaplnění biobanky,
	\item \textit{temperature\_data\_archive} -- archiv výstupů z teploměrů.
\end{compactitem}

Ve výchozím nastavení je součástí složky \textit{/home/bbmri\_data} také soubor \textit{bbmri\_logger.log} obsahující logovací záznamy aplikace. 

% ------------------------------------------------------------------------   
\subsection{Import a~zpracování dat}\label{chapter:implementation:subsection:import}
Jak již bylo zmíněno výše, data budou do systému automatizovaně nahrávána formou XML importů. Data jsou podle zdroje nebo kontextu nahrávána do složky příslušné biobanky a~podle typu dat je zvolena konkrétní podsložka. Např. pacientská data z~MOÚ (za předpokladu, že PostgreSQL přidělí MOÚ unikátní identifikátor $= 1$) budou importována do složky \textit{/bbmri\_data/biobank\_files/1/patient\_data}. Data jsou sice anonymizovaná, ale i~tak je vysoce nežádoucí, aby se dostala mimo systém. Z~toho důvodu byly pro přenos uvažovány pouze metody využívající šifrování, tj. protokol SCP (Secure Copy) nebo SFTP (SSH File Transfor Protocol). 

Aplikace obsahuje v~balíku cz.bbmri.trigeredEvents třídy, jejichž metody nejsou spustitelné z~webového rozhraní, ale jsou spouštěny pouze automaticky. Automatické spouštění je implementováno pomocí anotace \textit{@Scheduled}\footnote{org.springframework.scheduling.annotation.Scheduled}, která umožňuje definovat, kdy má být metoda spuštěna plánovačem úloh \textit{cron}. 
Pro ilustraci následující anotace spustí anotovanou metodu každý den deset minut po půlnoci. 
\begin{figure}[h!]
\centering
\begin{BVerbatim}
@Scheduled(cron = "10 0 * * * *")
\end{BVerbatim}
\end{figure}

Rutina pro import pacientů je nastavena tak, aby každý den minutu po půlnoci byly zkontrolovány adresáře všech biobank. Pokud adresář obsahuje nový soubor, tak je soubor přečten a~údaje jsou uloženy do databáze. Při úspěšném čtení dokumentu je XML soubor po vykonání rutiny přemístěn do složky \textit{patient\_data\_archive}, aby bylo možné zpětně analyzovat chování systému a~zároveň nebyly opakovaně zpracovávány stejné importy dat.
Obdobně jsou řešeny importy zaplnění i~naměřených dat, pouze s~rozdílem časů spuštění úloh.

Strukturu tříd zodpovídajících za zpracování XML importů popisuje UML diagram~\ref{fig:index:uml:class:parser}. Obě specializované třídy obsahují odkaz na konkrétní XML schéma (ve formátu XSD), takže jsou schopny provézt validaci dokument s~využitím metody abstraktní třídy. Dotazy na XML dokument jsou implementovány pomocí xPath.
Třída \textit{NamespaceContextMap} zajišťuje správné nastavení jmenných prostorů v~xPath dotazech.

Ve třídě \textit{MonitoringDataParser} je v~metodě \textit{getBoxPositions} použita třída \textit{PositionDTO} namísto \textit{Position} ukládané do databáze. Důvodem je, že parser nemá přístup do databáze, a~tak nemůže získat instanci třídy \textit{Sample} v~databázi uloženou podle zjištěného identifikátoru. Snazší a~přehlednější než vytvářet částečný objekt, jen s~identifikátorem instituce, je v~tomto případě předávat pouze načtený identifikátor. 

\begin{figure}[h!]
\centering
	\includegraphics[width=\textwidth]{ParserView}
\caption{UML class diagram popisující třídy pro zpracování XML importů.}
\label{fig:index:uml:class:parser}
\end{figure}

% ------------------------------------------------------------------------   
\subsection{Čištění a~kontrola dat}
Stejný mechanismus jako u~zpracování importů popsaný v~kapitole~\ref{chapter:implementation:subsection:import} je použit i~pro \uv{čistění} a~kontrolu dat. Ve stávající verzi aplikace je naplánovanou událostí řešena kontrola platnosti rezervací vzorků, viz~kapitola~\ref{fig:implementace:rezervace:cyklus}. 
Do budoucna bude tímto mechanismem řešeno i~mazání instancí dalších objektů, které již nejsou v~databázi potřebné. To ale bude záležet na požadavcích na délku archivace záznamů v~systému.

% ------------------------------------------------------------------------   
% ------------------------------------------------------------------------   
\section{Autorizace}
Implementace autorizace vychází z~příkladů popsaných v~knize~\cite{Stripes} a~je postavena na třídě \textit{InstanceBasedSecurityManager\footnote{org.stripesstuff.plugin.security}}. Tato třída umožňuje definovat oprávnění pomocí EL výrazů, což v praxi umožňuje definovat autorizaci na základě libovolné podmínky. V kombinaci s anotacemi (\textit{@PermitAll}, \textit{@DenyAll} a~\textit{@RolesAllowed}) definovanými v~\textit{javax.annotation.security} je tak možné specifikovat oprávnění velice podrobně. 

Oprávnění jsou definována na úrovni metod ActionBean tříd a~mohou být promítnuta i do JSP stránek díky \textit{security} tagům\footnote{http://www.stripes-stuff.org/security.tld}. Tagy \textit{<sec:allowed>} a \textit{<sec:notAllowed>} lze upravit obsah webové stránky pro autorizovaného i~neautorizovaného uživatele a při využití JSTL tagů\footnote{http://java.sun.com/jsp/jstl/core} i pro každou instanci relevantních objektů na stránce. 
Metoda s~omezením přístupu definovaným za pomoci EL výrazů může vypadat třeba následovně:

% Hack: mathescape - sice umozni zapis $, ale texnic to nepochopi a vznika neprehledny dokument
\begin{figure}[h!]
\centering
\begin{lstlisting}[mathescape=false]
@RolesAllowed({
	"biobank_operator if ${allowedBiobankVisitior}", "project_team_member if ${allowedProjectVisitor}"})
public Resolution foo() { 
				...
        return new ForwardResolution(FooActionBean.class);
    }
\end{lstlisting}
\end{figure}

Výše uvedený příklad popisuje, že operaci \textit{foo} může vykonat pouze uživatel s~rolí \textit{biobank\_operator}, který ale současně má oprávnění \textit{biobankEditor} k~aktuálně prohlížené biobance (definované předávaným parametrem v~URL) nebo \textit{project\_team\_member} s~oprávněním \textit{allowedProjectVisitor}.

Při vyhodnocování podmínky je pro uživatele s~rolí \textit{biobank\_operator} volána metoda označená \textit{allowedBiobankVisitor} . 

Je třeba upozornit na to, že kvůli použitým jmenným konvencím je ve skutečnosti volána metoda \textit{getAllowedBiobankVisitor()}. Tato metoda je definovaná v~abstraktní třídě \textit{PermissionActionBean}, kterou rozšiřuje většina ActionBean tříd.

\begin{figure}[h!]
\centering
\begin{lstlisting}
public boolean getAllowedBiobankVisitor() {
        return biobankAdministratorService.hasPermission(Permission.VISITOR, biobankId, getContext().getMyId());
    }
\end{lstlisting}
\end{figure}

% ------------------------------------------------------------------------   
% ------------------------------------------------------------------------   
\section{Formy informování uživatelů a~logování}
Systém s~uživatelem komunikuje na několika rovinách s~rozdílným cílem. Část informací slouží jen pro aktuální upozornění, část je ukládána nebo exportována mimo systém.

Základní komunikační nástroj mezi systémem a~uživatelem představují kontrolní nebo chybové výpisy. Ty jsou využívány pro informování uživatele o~právě vykonané operaci nebo o~chybě (např. validace), která při volané události nastala. Výpisy slouží pouze pro aktuálně přihlášeného uživatele a~nejsou nikam ukládány. Část chybových hlášení je ve stávající verzi vypisována uživateli. Důvodem je usnadnění procesu opravy při ladění systému. Pro testera, zkoušejícího na serveru nasazenou aplikaci, je efektivnější vidět okamžitý výpis na obrazovce, než při pádu aplikace dohledávat záznamy v~logu pád aplikace.

Pro běžného uživatele jsou připraveny tzv. notifikace (třída \textit{Notification} v~diagramu~\ref{fig:index:uml:class:user}), které upozorňují uživatele např. na změny provedené někým z~jeho kolegů nad společnými objekty (biobanka, projekt). Notifikace upozorní uživatele také na změnu stavu projektu nebo žádosti vyvolanou správcem biobanky. Ve stávající implementaci se tyto informace zobrazují na domovské stránce aplikace. Zobrazují se pouze nové (tj. označené jako nepřečtené) události. Při vytváření je zpráva přeložena do preferovaného jazyka uživatele.

Pro správce systému (\textit{administrator}) nebo vývojáře (\textit{developer}) je určena tzv. archivační služba. Při všech důležitých událostech v~systému je vygenerována zpráva o~vykonané události a~ta je uložena do databáze. Uživatel se zmíněným oprávněním může do výpisu těchto událostí nahlédnout. Zprávy jsou generovány jak pro události vyvolané uživatelem (schválení projektu), tak i~pro plánované událost spuštěné automaticky. Do archivační služby nejsou zapisována chybová hlášení.

Pro vývojáře (teď výjimečně ve smyslu osoba s~přístupem k~serveru, nikoli osoba s~touto systémovou rolí), je určeno logování. Pro logování byl použit aplikační rámec Logback\footnote{\url{http://logback.qos.ch/}}, který je implementací SLF4J API. Logování je zapisováno do souboru, aby bylo snadno dostupné. Cílem logování je zapisovat chybové (\textit{warn}, \textit{error}) a~nikoli informativní hlášení. Ve stávající implementaci jsou na několika místech ponechána i~informační hlášení pro jednodušší hledání chyb a~stopování běhu kódu před pádem. Tyto výpisy budou průběžně odstraňovány.


% ------------------------------------------------------------------------   
% ------------------------------------------------------------------------   
\section{Konfigurace}\label{chapter:implementation:section:configuration}
Chování aplikace je možné ovlivňovat na několika úrovních. Jak bylo popsáno výše, návrh a~implementace definuje systémové nastavení, ke kterému má přístup každý uživatel s~rolí \textit{administrátor}. Tato forma nastavení je určena především pro rozhodnutí spojená s~řízením a~koordinací projektu resp. biobanky. 

Další úroveň nastavení je definována pro každého samostatného uživatele. V~současné implementaci je touto cestou řešeno jen jazykové nastavení, ale do budoucna lze tímto způsobem ukládat preference uživatele týkající se chování nebo vzhledu aplikace (design, počet zobrazených položek na straně, automatické odhlášení, zasílání upomínek na e-mail, atd.).
Obě tyto formy nastavení jsou díky vazbě na databázi použitelné pro změnu chování aplikace za chodu. 

Nastavení aplikace z~pozice vývojáře naopak již součástí grafického rozhraní aplikace není, a~to jak z~praktických důvodů, tak mnohdy i~z~důvodů principiálních. Běžnou konvencí pro aplikace Javy EE a~související frameworky je nastavovat chování aplikace konfiguračními XML soubory. Pro konfiguraci aplikace jsou použity následující soubory:

\paragraph*{my.properties} -- konfigurační soubor výhradně pro chování samotné aplikace. Součástí souboru je nastavení \textit{StoragePath}, určující cestu pro uložení všech souborů popsaných v~\ref{chapter:implementation:subsection:organizationOfDataStorage}. Zde je třeba dát pozor při kopírování konfiguračního souboru mezi rozdílnými platformami. Pokud je aplikace provozována např. na linuxovém systému a~cesta ke složce byla zapsána ve formátu používaném v~OS Windows (tj. C:\textbackslash adresar\textbackslash druhy\_adresar), tak složka bude vytvořena zcela jinde než bylo zamýšleno. Nestandardní pojmenování může, kromě neočekávaného chování, způsobit, že složka nebude zobrazena programy jako např. WinSCP\footnote{http://winscp.net/}. 

Konfigurační soubor specifikuje také nutné údaje pro přístup k~databázi (jméno, heslo a název databáze).

\paragraph*{persistence.xml} -- soubor konfigurující chování aplikačního rámce Hibernate. Definuje, tzv. \textit{persistence unit}, což je kolekce entit, které mají být spravovány \textit{EntityManagerem}.
Konfigurační soubor definuje dvě instance \textit{persistence unit}. První je pro běžný chod aplikace a~druhá, nazvaná \textit{TestPersistenceUnit}, je pro testování a~využívá databáze uložené pouze v~paměti.
Konkrétně byla použita implementace HSQLDB\footnote{\url{http://hsqldb.org/}} (HyperSQL DataBase).

\paragraph*{logback-test.xml} -- definuje pravidla logování aplikace. Pro logování byl použit aplikační rámec Logback. Logovací zprávy jsou zapisovány do souboru \textit{bbmri\_logger.log}. Při kopírování konfiguračního souboru mezi platformami je třeba dát pozor na správné stanovení cesty k~souboru.

\paragraph*{test-applicationContext.xml} -- definuje aplikační kontext pro framework Spring pro potřeby testů. Kontext je vázán na \textit{TestPersistenceUnit} popsaný výše.

\paragraph*{spring-context.xml} -- aplikační kontext pro rámec Spring. Mimo jiné zajišťuje, že nastavení z~konfiguračního souboru \textit{my.properties} jsou v~aplikaci zohledněna.

\paragraph*{pom.xml} -- konfigurační soubor pro nástroj Maven, popisující, jaké programové nástroje (knihovny, frameworky) jsou aplikací využívány. Některé knihovny nejsou součástí Mavenu a~musely být staženy manuálně (např. Stripes). JavaScriptové knihovny nejsou součástí Mavenu.

\paragraph*{web.xml} -- konfigurační soubor webových aplikací Javy EE. Definuje napojení aplikace na rámec Stripes, kódování aplikace, jaké jazyky jsou podporovány v~překladech, kde jsou překlady uloženy a~další nastavení ovlivňující prezentační vrstvu.

Nejnižší úrovní nastavení je konfigurace samotného serveru a~SW nezbytného pro chod aplikace. Seznam programů nutných pro běh serveru se nachází v příloze~\ref{appendix:server}.

% ------------------------------------------------------------------------   
% ------------------------------------------------------------------------   
\section{Práva k~využívání převzatých částí kódu}
V~tabulce~\ref{tab:libraries:license} jsou uvedeny licence použitých částí kódu.
Všechny uvedené programové nástroje jsou k~dispozici pod některou z~variant OpenSource licencí \cite{Licence}.

\begin{table}[ht] 
\centering
\begin{tabular}{l l}
\hline 
Knihovna & Licence \\
\hline \hline
Hibernate 									& LGPL 2.1 \\
Logback 										& LGPL 2.1 \\
Stripesstuff 								& Apache License 2.0 \\ 
Stripes 										& Apache License 2.0 \\ 
Spring 											& Apache License 2.0 \\ 
com.google.guava 						& Apache License 2.0 \\
commons-lang 								& Apache License 2.0 \\
commons-io 									& Apache License 2.0 \\
org.apache.ws.commons.axiom & Apache License 2.0 \\
stax-api 										& Apache License 2.0 \\
junit 											& Common Public License Version 1.0 \\
org.hsqldb 									& BSD \\
org.mockito 								& The MIT License \\
bootstrap 									& The MIT License \\
Flot 												& The MIT License \\

\hline %inserts single line 
\end{tabular} 
\caption{Licence použitých frameworků, knihoven a~částí kódu.}
\label{tab:libraries:license} % is used to refer this table in the text 
\end{table} 

% ------------------------------------------------------------------------   
% ------------------------------------------------------------------------   
\section{Testování}
Aplikace byla testována na několika úrovních, různými způsoby. Pro entity obsahující složitější metody, které bylo nezbytné testovat, byly implementovány jednoduché jednotkové testy. 

Druhou testovanou úrovní byly testy DAO tříd. Ty slouží např. k~otestování správnosti anotací entit (anotace pro Hibernate definující kardinalitu vztahů) nebo k~otestování dotazů kladených na databázi. Pro implementaci těchto testů je použit nový konfigurační soubor pro Spring, který pracuje s~databází drženou v~paměti tak, aby nebylo nutné vracet skutečnou databázi po testech do původního stavu. Jako databáze byla použita HSQLDB (HyperSQL DataBase).

Další fází testování bylo ověření funkce aplikační (servisní) vrstvy. Aby bylo možné testovat pouze funkcionalitu aplikační vrstvy bez vlivu na nižší závislosti, jsou v~testech použity tzv. \textit{mock} objekty. Ty slouží k~simulaci chování DAO vrstvy, takže testy ověřující výhradně chování aplikační vrstvy.

Pro manuální testování webového rozhraní byly používány lokální uživatelské účty, aby bylo možné se k~aplikaci přihlásit bez konfigurace lokálního IdP pro Shibboleth. Tento způsob autentizace uživatele není při ostrém chodu aplikace vůbec povolen.


% ------------------------------------------------------------------------   
% ------------------------------------------------------------------------      
% Závěr
% ------------------------------------------------------------------------   
\chapter{Závěr}
Diplomová práce popisuje realizaci informatické části projektu \ProjectName. Moje role v~projektu spočívala v~převzetí vstupní analýzy z~roku 2011~\cite{ARCH_2011_12_29}, na jejímž základě jsem vytvořil prototyp aplikace. Prototyp byl konzultován s~lékařskými i~informatickými pracovníky z~vedení projektu i~ze zapojených institucí. Průběžným výstupem z~jednání a~prezentací byly požadavky, co je třeba upravit a~změnit, jak po stránce work-flow aplikace, tak i~v~datovém modelu. Mým úkolem bylo na základě těchto požadavků aktualizovat exportní schémata a~nové skutečnosti doplňovat do původního dokumentu. Ve spolupráci s~RNDr. Petrem Holubem, Ph.D. vznikla aktualizovaná verze vstupní analýzy, která popisuje revidované požadavky na informatickou infrastrukturu. Můj vklad do projektu tedy spočíval především v~návrhu a~vývoji popsané aplikace, ale podílel jsem se i~na podobě vstupní analýzy. Část požadavků vzešla z~výsledků mých jednání a~mé oponentury původního návrhu infrastruktury.

Průběh vývoje by se v~teminologii SW vývoje dal nejlépe popsat jako iterativní. Důvodem pro iterace byly mé postupně rostoucí zkušenosti s~vývojem Java EE aplikace i~měnící se požadavky ze strany vedení projektu na strukturu dat a~způsob interakce oprávněných osob s~indexem biobank. V~procesu vývoje došlo několikrát k~výraznému refaktoringu, což dokládají statistiky repozitáře použitého pro vývoj\footnote{\url{https://github.com/Ondrej-vojtisek/bbmri/graphs/contributors}}.

Aplikace byla implementována v~rozsahu, jaký popisuje ~\ref{chapter:implementation}. kapitola, a~je dostupná na uvedené URL. 
Aplikace (stav ke konci května 2014) je spouštěna a~dostupná, ale zatím k~ní nebyly připojeny exportní moduly ze spolupracujících institucí. Dle posledních informací je exportní moduly na MOÚ a~na 1.~LF ve fázi finalizace. V~podobném stavu je fáze spolupráce se společností KESA. Je známo jak data předávat, je jasný formát dat, chybí jen finální orchestrace. 

Jedním z~možných směrů budoucího vývoje je úprava webového rozhraní tak, aby bylo možné odesílat dílčí změny webových formulářů bez nutnosti znovu načítat celou webovou stránku. AJAX je v~rámci Stripes podporován, takže rozšíření tímto směrem není z~technologického hlediska problém. Další možné zlepšení by bylo např. využití SVG elementů pro názornější zobrazení prvků infrastruktury biobank. Kromě těchto plánovaných spíše kosmetických úprav lze očekávat další revize požadavků s~prvními uživateli aplikace.

%% Lists of tables and figures, glossary, etc.
%\printindex
%\printglossary
%\listoffigures
%\listoftables

%% Bibliography from references.bib
%\begingroup
%\def\tmpchapter{0}
%\renewcommand{\chaptername}{}
%\renewcommand{\thechapter}{}
%\addtocontents{toc}{\setcounter{tocdepth}{-1}}
%\chapter{Zdroje}
%\renewcommand{\chapter}[2]{}% for other classes

% Následují další kapitoly a podkapitoly, popřípadě závěr, dodatky, 
% seznam literatury či použitých obrázků nebo tabulek.

% LaTeX magic :) Při kopírování některé znaky chybí !!!!!
\renewcommand{\UrlBreaks}{\do\/\do\a\do\b\do\c\do\e\do\f\do\i\do\j\do\k\do\l\do\m\do\n\do\o\do\q\do\r\do\s\do\u\do\v\do\w\do\x\do\y\do\z\do\A\do\B\do\C\do\D\do\E\do\F\do\G\do\H\do\I\do\J\do\K\do\L\do\M\do\N\do\O\do\P\do\Q\do\R\do\S\do\T\do\U\do\V\do\W\do\X\do\Y\do\Z\do\?\do\=\do\-\do\0\do\1\do\3\do\4\do\5\do\6\do\7\do\8\do\9\do\p\do\d\do\h}
\mathchardef\UrlBreakPenalty=50\relax

\clearpage
\addcontentsline{toc}{chapter}{\textbf{Literatura}}
\bibliographystyle{plainnat} %./splncs}
\bibliography{references}
%\endgroup

% Prostředí pro přílohy
\begin{appendix}

% ------------------------------------------------------------------------   
% ------------------------------------------------------------------------  
\chapter{Exportní schémata}
% ------------------------------------------------------------------------  
Všechna exportní schémata jsou definována ve formátu RelaxNG Compact.

\begin{figure}[htbp]
\centering
\begin{lstlisting}[language=XML, caption={Element popisující pacienta v~exportním schématu.}, label={fig:export:data:patient}]
element patient {		
	attribute id { xsd:string { maxLength = "10" } },		
	attribute consent { xsd:boolean },   
	attribute year { xsd:gYear },	  
	attribute month { xsd:gMonth },   
	attribute sex { "male" | "female" },	  
	biobankId,		
	element LTS { ( tissue | serum | genome | diagnosisMaterial )* },		
	element STS { ( diagnosisMaterial )* }
}

\end{lstlisting}
\end{figure}

\begin{figure}[htbp]
\centering
\begin{lstlisting}[language=XML, caption={Element popisující tkáň v~exportním schématu.},
label={fig:export:data:tissue}]
element tissue {
	attribute year { xsd:gYear },
	attribute number { xsd:string { maxLength = "6" } },
	attribute sampleId { xsd:string { maxLength = "32" } },
	element samplesNo { xsd:int },
	element availableSamplesNo { xsd:int },
	element materialType { xsd:string { maxLength = "4" } },
	element TNM { xsd:string { maxLength = "7" pattern = "[a-zA-Z0-9]+" } },
	element pTNM { xsd:string { maxLength = "7" pattern = "[a-zA-Z0-9]+" } },
	( element morphology { xsd:string { length = "7" pattern = "[0-9]{4}/[0-9]{2}"} } | element grading { xsd:int { minInclusive = "1" maxInclusive = "9" } } ),
	element cutTime { xsd:dateTime },
	element freezeTime { xsd:dateTime },
	element retrieved { "preoperational" | "operational" | "post" | "unknown" }
}
\end{lstlisting}
\end{figure}

\begin{figure}[htbp]
\centering
\begin{lstlisting}[language=XML, caption={Element popisující sérum v~exportním schématu.},
label={fig:export:data:serum}]
element serum {
	attribute year { xsd:gYear },
	attribute number { xsd:string { maxLength = "6" } },
	attribute sampleId { xsd:string { maxLength = "32" } },
	element samplesNo { xsd:int },
	element availableSamplesNo { xsd:int },
	element materialType { xsd:string { maxLength = "4" } }
	element takingDate { xsd:dateTime },
	element retrieved { "preoperational" | "operational" | "post" | "unknown" }
}
\end{lstlisting}
\end{figure} 

\begin{figure}[htbp]
\centering
\begin{lstlisting}[language=XML, caption={Element popisující genomovou krev v~exportním schématu.},
label={fig:export:data:genome}]
element genome {
	attribute year { xsd:gYear },
	attribute number { xsd:string { maxLength = "6" } },
	attribute sampleId { xsd:string { maxLength = "32" } },
	element samplesNo { xsd:int },
	element availableSamplesNo { xsd:int },
	element materialType { xsd:string { maxLength = "4" } }
	element takingDate { xsd:dateTime },
	element retrieved { "preoperational" | "operational" | "post" | "unknown" }
}
\end{lstlisting}
\end{figure}

\begin{figure}[htbp]
\centering
\begin{lstlisting}[language=XML, caption={Element popisující materiál se stanovenou diagnózou v~exportním schématu.}, label={fig:export:data:diagnosisMaterial}]
element diagnosisMaterial {
	attribute year { xsd:gYear },
	attribute number { xsd:string { maxLength = "6" } },
	attribute sampleId { xsd:string { maxLength = "32" } },
	element materialType { xsd:string { maxLength = "4" } }
	element diagnosis { xsd:string { pattern = "[a-zA-Z0-9]+" minLength = "3" maxLength = "5" } },
	element takingDate { xsd:dateTime },
	element retrieved { "preoperational" | "operational" | "post" | "unknown" }
}
\end{lstlisting}
\end{figure}

% ------------------------------------------------------------------------   
% ------------------------------------------------------------------------   
\chapter{Konfigurace serveru}\label{appendix:server}
% ------------------------------------------------------------------------   

\section{Seznam instalovaného SW}
\begin{itemize}
	\item Apache -- webový server
	\item Shibboleth (apache2-mod-shib2) -- nástroj pro integraci do federace eduId
	\item PostgreSQL -- databáze
	\item Tomcat -- aplikační server
	\item Apache2-mpm-prefork -- doplněk do Apache pro zajištění kompatibility s~knihovnami nepodporujícími vlákna
	\item Cronolog -- logování ze standardního výstupu aplikačního serveru
	\item Java 
	
\end{itemize}

% ------------------------------------------------------------------------   
% ------------------------------------------------------------------------   
\chapter{Číselníky jednotlivých institucí}
% ------------------------------------------------------------------------   

\section{MOÚ}

\begin{table}[h!] 
\centering
\begin{tabular}{l l l l l l}
\hline 
Klíč & Název 	& Prim. mat. &  \ProjectName \\ [0.5ex]  
%heading 
\hline \hline
1		&	Nádor maligní							&	T,TDC		&			tissue/01 \\
2		&	Metastáza									&	T,TDC		&			tissue/02 \\
3		&	Nádor benigní							&	T,TDC		&			tissue/03 \\
4		&	Zdravá tkáň								&	T,TDC		&			tissue/04 \\
5		&	Premaligní tkáň						&	T,TDC		&			tissue/04 \\
53	&	Maligní-p53 (RNA-LATER)		&	RNA			&			tissue/01 \\
54	&	Zdravá-p53 (RNA LATER)		&	RNA			&			tissue/04 \\
55	&	Metastáza-p53(RNA LATER)	&	RNA			&			tissue/02 \\
56	&	Benigní-p53 (RNA-LATER)		&	RNA			&			tissue/03 \\
6		&	Inflamatorní tkáň					&	T,TDC		&			tissue/04 \\
A1	&	Maligní t.-výzkum Cha			&	v-Ch		&			tissue/01 \\
A4	&	Zdravá t.-výzkum Cha			&	v-Ch		&			tissue/04 \\
B1	&	Maligní t.-výzkum Gastro	&	v-G			&			tissue/01 \\
B4	&	Zdravá t.-výzkum Gastro		&	v-G			&			tissue/04 \\
gD	&	Genomová DNA							&	gDNA		&			genome/gD \\
N		&	Nemaligní punktát					&	PSD			&							  \\	
NE	&	Maligní tkáň-NEO ALTTO		&	T,TDC		&			tissue/01 \\
P		&	Maligní punktát						&	PSD			&			 					\\
PD	&	Plasma dusík							&	PSD,PSDC&			 					\\
PK	&	Plná krev									&	PK			&			genome/PK \\
Pl	&	Plasma										&	PS			&		 	 				  \\
S~	&	Sérum											&	PS			&		  blood/SD 	\\
SD	&	Sérum dusík								&	PSD,PSDC&		  blood/SD 	\\
SE	&	Maligní tkáň-SELDI				&	T,TDC		&			tissue/01 \\
U1	&	Maligní tkáň UK						&	UK			&			tissue/01 \\
U2	&	Metastáza UK							&	UK			&			tissue/02 \\
U4	&	Zdravá tkáň UK						&	UK			&			tissue/04 \\

\hline %inserts single line 
\end{tabular} 
\caption{Číselník materiálů využívaný na MOÚ~\cite{ARCH_2014_1_25}.}
\label{tab:ciselnik-mat-muni}
\end{table} 
\newpage

\section{LF~UP~}
\begin{multicols}{2}
\textbf{Pacient, odběr~\cite{ARCH_2014_1_25}:} 
\begin{compactitem}
	\item Buňky 
		\begin{compactitem}
			\item BD, BE, LB, BO, BS, BU
			\item BZ, LT, TR, KB, KK, PR
		\end{compactitem}

	\item Dusík 
		\begin{compactitem}
			\item TD
		\end{compactitem}

	\item DNA 
		\begin{compactitem}
			\item DG, DK, DM, DW
		\end{compactitem}

	\item Amplikon 
		\begin{compactitem}
			\item DA, DP
		\end{compactitem}
	
	\item Plazma/sérum 
		\begin{compactitem}
			\item KE, KP
		\end{compactitem}
	
	\item RNA 
		\begin{compactitem}
			\item RC, RV
		\end{compactitem}

	\item Řezy 
		\begin{compactitem}
			\item PP
		\end{compactitem}

	\item Sklíčka RT 
		\begin{compactitem}
			\item PS
		\end{compactitem}

	\item Sklíčka zamražená 
		\begin{compactitem}
			\item RY, NC, NR, BC
		\end{compactitem}
 \end{compactitem}

\end{multicols}


\begin{table}[h!] 
\centering
\begin{tabular}{l l}
\hline 
Typ materiálu & Kód \\
\hline \hline
Buněčná suspenze 									& BS \\
DNA genomická 										& DG \\
DNA komplementární 								& DK \\
DNA whole genome amplified 				& DW \\
Kostní dřeň nativní a~nesrážlivá 	& RN \\
Krev nesrážlivá 									& KN \\
Krev plazma 											& KP \\
Krev sérum 												& KE \\
Krev srážlivá 										& KA \\
Laváž 														& LA \\
Likvor 														& LI \\
Moč 															& MO \\
Nátěr buněčný 										& NB \\
Nátěr chromozomální 							& NC \\
Nátěr kostní dřeň 								& NK \\
Šťáva pankreatická  							& SP \\
Parafinový blok 									& PB \\
Parafinový řez 										& PR \\
RNA 															& RN \\
Tkáň nativní 											& TN \\
Tkáň RNA later 										& TR \\
Tkáň zamražená 										& TZ \\
Výpotek 													& VY \\

\hline %inserts single line 
\end{tabular} 
\caption{Číselník materiálů využívaný biobankou LF~UP~\cite{ARCH_2014_1_25}.}
\label{tab:ciselnik-mat-LFUP} % is used to refer this table in the text 
\end{table} 

\newpage

\section{1.~LF}

\begin{table}[h!] 
\centering
\begin{tabular}{l l}
\hline 
Typ materiálu & Kód \\
\hline \hline
Nádor maligní 							& Tm 	\\
Metastáza 									& Te 	\\
Nádor benigní 							& Tb 	\\
Zdravá tkáň 								& Th 	\\
Patol. tk. 									& Tp 	\\
Nádor nejisté biol. povahy 	& Tn 	\\
Carcinoma in situ 					& Ti 	\\
Suffix pro later 						& -L 	\\
Sérum 											& S~	\\
Plazma 											& P 	\\
Genomová DNA 								& G 	\\
Plná krev 									& B 	\\
Moč 												& U~	\\

\hline %inserts single line 
\end{tabular} 
\caption{Číselník materiálů využívaný biobankou 1.~LF~UK~\cite{ARCH_2014_1_25}.}
\label{tab:ciselnik-mat-Ilfuk} % is used to refer this table in the text 
\end{table} 

\end{appendix}


%% End of the whole document
\end{document}